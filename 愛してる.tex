% !TeX encoding = UTF-8
% !TeX program = LuaLaTeX

%\documentclass[12pt]{article}
\documentclass[14pt]{extreport}
% !TeX encoding = UTF-8
% !TeX program = LuaLaTeX

% !TeX encoding = UTF-8
% !TeX program = LuaLaTeX

%\documentclass[12pt]{article}

%\usepackage{luatexja-preset}
%\usepackage{fontspec}
\usepackage[no-math]{luatexja-fontspec}
\usepackage{luatexja-ruby}
\usepackage{color}

\topmargin=-0.95in      %
\evensidemargin=0in     %
\oddsidemargin=0in      %
\textwidth=6.5in        %
\textheight=9.75in       %
\headsep=0.25in

\setmainfont{Noto Sans CJK JP}
\setmainjfont{Noto Sans CJK JP}

\makeatletter
\def\dynscriptsize{\check@mathfonts\fontsize{\sf@size}{\z@}\selectfont}
\makeatother
\def\textunderset#1#2{\leavevmode
  \vtop{\offinterlineskip\halign{%
    \hfil##\hfil\cr\strut#2\cr\noalign{\kern-.3ex}
    \hidewidth\dynscriptsize\strut#1\hidewidth\cr}}}
\newcommand{\righttext}[1]{\ifx&#1&\else
  \hspace{1em} \unskip
  \nobreak \hfil \penalty1000 \hfilneg \indent
  \strut\hfill \mbox{\def\z{\zhuyin}\small #1}
  \fi}

\newcommand{\zhuyin}[2]{\ruby{#1}{#2}}
\newcommand{\kanji}[2]{\textunderset{#1}{\textcolor{blue}{#2}}}
\def\jisho{\righttext}
\def\trans{\righttext}

\newcommand{\lyrics}[1]{\begingroup
\def\z{\zhuyin}
\list{}{\leftmargin0em}
\setlength\itemsep{1em}
  #1
\endlist
\endgroup}

\makeatletter
\newcommand{\itemmark}[2]{%
\item[#1.]
\def\@currentlabel{#1}\label{par:#1}
  \expandafter\def\csname firstline@#1\endcsname{#2}%
  #2
}
\newcommand{\itemrepeat}[1]{
  \item[\ref{par:#1}.]
  \csname firstline@#1\endcsname
}
\makeatother


\renewcommand{\jisho}[1]{}

\renewcommand{\lyrics}[1]{\begingroup
\def\z{\zhuyin}
\list{}{\leftmargin0em}
\setlength\itemsep{1em}
  #1
\endlist
\endgroup
\clearpage
\begingroup
\def\z{\kanji}
\list{}{\leftmargin0em}
\setlength\itemsep{1em}
  #1
\endlist
\endgroup}

\begin{document}

\lyrics{
\item
  \textbf{\zhuyin{愛}{あい}してる \hfill%
    「\zhuyin{続}{ぞく} \zhuyin{夏|目}{なつ|め}\zhuyin{友|人|帳}{ゆう|じん|ちょう}」のED}

\item
  ねぇ もう\z{少}{すこ}しだけ
  \jisho{}

  もう\z{少}{すこ}しだけ \z{聞}{き}いていてほしい
  \jisho{}

  ねぇ もう\z{少}{すこ}しだけ
  \jisho{}

  もう\z{少}{すこ}しだけ わがままいいですか?
  \jisho{}

\item
  \z{手}{て}にいれた\z{途端}{とたん}に \z{消}{き}えてしまいそう
  \jisho{}

  \z{言葉}{ことば}をくれませんか?
  \jisho{}

\item
  『\z{愛}{あい}している \z{愛}{あい}している \z{世界}{せかい}が\z{終}{お}わるまで』
  \jisho{}

  \z{馬鹿}{ばか}げてると\z{笑}{わら}いながら \z{口}{くち}に\z{出}{だ}してみて
  \jisho{}

  \z{愛}{あい}している
  \jisho{}

  そんなことが\z{簡単}{かんたん}には\z{出来}{でき}なくて
  \jisho{}

  \z{上手}{うま}く\z{愛}{あい}せるようにと
  \jisho{}

  あの\z{空}{そら}に\z{祈}{いの}っている
  \jisho{}

\item
  ねぇ \z{知}{し}りたくても
  \jisho{}

  \z{知}{し}り\z{尽}{つ}くせないことばかりで
  \jisho{}

  だから \z{1}{ひと}つにならない\z{2}{ふた}つの\z{体}{からだ}を
  \jisho{}

  \z{精一杯}{せいいっぱい} \z{抱}{だ}きしめた
  \jisho{}

\item
  あなたがいるそれだけで もう\z{世界}{せかい}が\z{変}{か}わってしまう
  \jisho{}

  モノトーンの\z{景色}{けしき}が ほら\z{鮮}{あざ}やかに\z{映}{うつ}る
  \jisho{}

  いつの\z{間}{ま}にか\z{離}{はな}れていた \z{手}{て}をつないで\z{歩}{ある}いてく
  \jisho{}

  \z{上手}{うま}く\z{愛}{あい}せているかなぁ
  \jisho{}

  あの\z{空}{そら}に\z{聞}{き}いてみるの
  \jisho{}

\item
  いつか\z{離}{はな}ればなれになる\z{日}{ひ}がきても
  \jisho{}

  あなたを\z{想}{おも}った\z{日々}{ひび}があればそれでいい
  \jisho{}

  いつか\z{離}{はな}れた\z{意味}{いみ}を\z{知}{し}る\z{日}{ひ}が\z{来}{く}るよ
  \jisho{}

  \z{約束}{やくそく}するから \z{明日}{あした}へ
  \jisho{}

\item
  『\z{愛}{あい}している \z{愛}{あい}している \z{世界}{せかい}が\z{終}{お}わるまで』
  \jisho{}

  \z{馬鹿}{ばか}げてると\z{笑}{わら}いながら \z{口}{くち}に\z{出}{だ}してみて
  \jisho{}

  \z{愛}{あい}している
  \jisho{}

  そんなことが\z{簡単}{かんたん}には\z{出来}{でき}なくて
  \jisho{}

  \z{上手}{うま}く\z{愛}{あい}せるようにと
  \jisho{}

  あの\z{空}{そら}に\z{祈}{いの}っている
  \jisho{}

\item
  あの\z{空}{そら}に\z{祈}{いの}っている
  \jisho{}


}

\end{document}

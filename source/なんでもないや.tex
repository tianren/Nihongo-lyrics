% !TeX encoding = UTF-8
% !TeX program = LuaLaTeX

%\documentclass[12pt]{article}
\documentclass[14pt]{ltjsarticle}
../macros/dual-lyrics-luamacros.tex
\begin{document}

\lyrics{
\item
  \textbf{なんでもないや \hfill %
    「君の名は。」の\zhuyin{主}{しゅ}\zhuyin{題}{だい}\zhuyin{歌}{か}}

\item
  \z{二人}{ふたり}の\z{間}{あいだ} \z{通}{とお}り\z{過}{す}ぎた\z{風}{かぜ}は
  \jisho{}

  どこから\z{寂}{さび}しさを\z{運}{はこ}んできたの
  \jisho{}

  \z{泣}{な}いたりした そのあとの\z{空}{そら}は
  \jisho{}

  やけに\z{透}{す}き\z{通}{とお}っていたりしたんだ
  \jisho{}

\item
  いつもは\z{尖}{とが}ってた\z{父}{ちち}の\z{言葉}{ことば}が
  \jisho{}

  \z{今日}{きょう}は\z{暖}{あたた}かく\z{感}{かん}じました
  \jisho{}

  \z{優}{やさ}しさも\z{笑顔}{えがお}も\z{夢}{ゆめ}の\z{語}{かた}り\z{方}{かた}も
  \jisho{}

  \z{知}{し}らなくて\z{全部}{ぜんぶ} \z{君}{きみ}を\z{真似}{まね}たよ
  \jisho{}

\item
  もう\z{少}{すこ}しだけでいい あと\z{少}{すこ}しだけでいい
  \jisho{}

  もう\z{少}{すこ}しだけでいいから
  \jisho{}

  もう\z{少}{すこ}しだけでいい あと\z{少}{すこ}しだけでいい
  \jisho{}

  もう\z{少}{すこ}しだけ くっついていようか
  \jisho{}

\item
  \z{僕}{ぼく}らタイムフライヤー
  \jisho{}

  \z{時}{とき}を\z{駆}{か}け\z{上}{あ}がるクライマー
  \jisho{}

  \z{時}{とき}のかくれんぼ はぐれっこは もういやなんだ
  \jisho{}

\item
  \z{嬉}{うれ}しくて\z{泣}{な}くのは \z{悲}{かな}しくて\z{笑}{わら}うのは
  \jisho{}

  \z{君}{きみ}の\z{心}{こころ}が \z{君}{きみ}を\z{追}{お}い\z{越}{こ}したんだよ
  \jisho{}

\item
  \z{星}{ほし}にまで\z{願}{ねが}って \z{手}{て}にいれたオモチャも
  \jisho{}

  \z{部屋}{へや}の\z{隅}{すみ}っこに\z{今}{いま} \z{転}{ころ}がってる
  \jisho{}

  \z{叶}{かな}えたい\z{夢}{ゆめ}も \z{今日}{きょう}で100\z{個}{こ}できたよ
  \jisho{}

  たった\z{一}{ひと}つと いつか \z{交換}{こうかん}こしよう
  \jisho{}

\item
  いつもは\z{喋}{しゃべ}らない あの\z{子}{こ}に\z{今日}{きょう}は
  \jisho{}

  \z{放課後}{ほうかご}「また\z{明日}{あした}」と\z{声}{こえ}をかけた
  \jisho{}

  \z{慣}{な}れないことも たまにならいいね
  \jisho{}

  \z{特}{とく}にあなたが \z{隣}{となり}にいたら
  \jisho{}

\item
  もう\z{少}{すこ}しだけでいい あと\z{少}{すこ}しだけでいい
  \jisho{}

  もう\z{少}{すこ}しだけでいいから
  \jisho{}

  もう\z{少}{すこ}しだけでいい あと\z{少}{すこ}しだけでいい
  \jisho{}

  もう\z{少}{すこ}しだけ くっついていようよ
  \jisho{}

\item
  \z{僕}{ぼく}らタイムフライヤー \z{君}{きみ}を\z{知}{し}っていたんだ
  \jisho{}

  \z{僕}{ぼく}が \z{僕}{ぼく}の\z{名前}{なまえ}を \z{覚}{おぼ}えるよりずっと\z{前}{まえ}に
  \jisho{}

\item
  \z{君}{きみ}のいない \z{世界}{せかい}にも
  \jisho{}

  \z{何}{なに}かの\z{意味}{いみ}はきっとあって
  \jisho{}

  でも \z{君}{きみ}のいない \z{世界}{せかい}など
  \jisho{}

  \z{夏休}{なつやす}みのない \z{八月}{はちがつ}のよう
  \jisho{}

  \z{君}{きみ}のいない \z{世界}{せかい}など
  \jisho{}

  \z{笑}{わら}うことない サンタのよう
  \jisho{}

  \z{君}{きみ}のいない \z{世界}{せかい}など
  \jisho{}

\item
  \z{僕}{ぼく}らタイムフライヤー
  \jisho{}

  \z{時}{とき}を\z{駆}{か}け\z{上}{あ}がるクライマー
  \jisho{}

  \z{時}{とき}のかくれんぼ はぐれっこは もういやなんだ
  \jisho{}

\item
  なんでもないや やっぱりなんでもないや
  \jisho{}

  \z{今}{いま}から\z{行}{い}くよ
  \jisho{}

\item
  \z{僕}{ぼく}らタイムフライヤー
  \jisho{}

  \z{時}{とき}を\z{駆}{か}け\z{上}{あ}がるクライマー
  \jisho{}

  \z{時}{とき}のかくれんぼ はぐれっこは もういいよ
  \jisho{}

\item
  \z{君}{きみ}は\z{派手}{はで}なクライヤー その\z{涙}{なみだ} \z{止}{と}めてみたいな
  \jisho{}

  だけど \z{君}{きみ}は\z{拒}{こば}んだ
  \jisho{}

  \z{零}{こぼ}れるままの\z{涙}{なみだ}を\z{見}{み}てわかった
  \jisho{}

\item
  \z{嬉}{うれ}しくて\z{泣}{な}くのは \z{悲}{かな}しくて \z{笑}{わら}うのは
  \jisho{}

  \z{僕}{ぼく}の\z{心}{こころ}が \z{僕}{ぼく}を\z{追}{お}い\z{越}{こ}したんだよ
  \jisho{}
}
\end{document}

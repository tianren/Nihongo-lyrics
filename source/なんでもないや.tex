% !TeX encoding = UTF-8
% !TeX program = LuaLaTeX

%\documentclass[12pt]{article}
\documentclass[14pt]{ltjsarticle}
% !TeX encoding = UTF-8
% !TeX program = LuaLaTeX

% !TeX encoding = UTF-8
% !TeX program = LuaLaTeX

%\documentclass[12pt]{article}

%\usepackage{luatexja-preset}
%\usepackage{fontspec}
\usepackage[no-math]{luatexja-fontspec}
\usepackage{luatexja-ruby}
\usepackage{color}

\topmargin=-0.95in      %
\evensidemargin=0in     %
\oddsidemargin=0in      %
\textwidth=6.5in        %
\textheight=9.75in       %
\headsep=0.25in

\setmainfont{Noto Sans CJK JP}
\setmainjfont{Noto Sans CJK JP}

\makeatletter
\def\dynscriptsize{\check@mathfonts\fontsize{\sf@size}{\z@}\selectfont}
\makeatother
\def\textunderset#1#2{\leavevmode
  \vtop{\offinterlineskip\halign{%
    \hfil##\hfil\cr\strut#2\cr\noalign{\kern-.3ex}
    \hidewidth\dynscriptsize\strut#1\hidewidth\cr}}}
\newcommand{\righttext}[1]{\ifx&#1&\else
  \hspace{1em} \unskip
  \nobreak \hfil \penalty1000 \hfilneg \indent
  \strut\hfill \mbox{\def\z{\zhuyin}\small #1}
  \fi}

\newcommand{\zhuyin}[2]{\ruby{#1}{#2}}
\newcommand{\kanji}[2]{\textunderset{#1}{\textcolor{blue}{#2}}}
\def\jisho{\righttext}
\def\trans{\righttext}

\newcommand{\lyrics}[1]{\begingroup
\def\z{\zhuyin}
\list{}{\leftmargin0em}
\setlength\itemsep{1em}
  #1
\endlist
\endgroup}

\makeatletter
\newcommand{\itemmark}[2]{%
\item[#1.]
\def\@currentlabel{#1}\label{par:#1}
  \expandafter\def\csname firstline@#1\endcsname{#2}%
  #2
}
\newcommand{\itemrepeat}[1]{
  \item[\ref{par:#1}.]
  \csname firstline@#1\endcsname
}
\makeatother


\renewcommand{\jisho}[1]{}

\renewcommand{\lyrics}[1]{\begingroup
\def\z{\zhuyin}
\list{}{\leftmargin0em}
\setlength\itemsep{1em}
  #1
\endlist
\endgroup
\clearpage
\begingroup
\def\z{\kanji}
\list{}{\leftmargin0em}
\setlength\itemsep{1em}
  #1
\endlist
\endgroup}

\begin{document}

\lyrics{
\item
  \textbf{なんでもないや \hfill %
    「君の名は」のOP/ED/\zhuyin{主}{しゅ}\zhuyin{題}{だい}\zhuyin{歌}{か}}

\item
  \z{二人}{ふたり}の\z{間}{あいだ} \z{通}{とお}り\z{過}{す}ぎた\z{風}{かぜ}は
  \jisho{}

  
どこから\z{寂}{さび}しさを\z{運}{はこ}んできたの
  \jisho{}

  \z{泣}{な}いたりしたそのあとの\z{空}{そら}は
  \jisho{}

  
やけに\z{透}{す}き\z{通}{とお}っていたりしたんだ
  \jisho{}

\item
  
いつもは\z{尖}{とが}ってた\z{父}{ちち}の\z{言葉}{ことば}が
  \jisho{}

  \z{今日}{きょう}は\z{暖}{あたた}かく\z{感}{かん}じました
  \jisho{}

  \z{優}{やさ}しさも\z{笑顔}{えがお}も\z{夢}{ゆめ}の\z{語}{かた}り\z{方}{かた}も
  \jisho{}

  \z{知}{し}らなくて\z{全部}{ぜんぶ} \z{君}{きみ}を\z{真似}{まね}たよ
  \jisho{}

\item
  
もう\z{少}{すこ}しだけでいい あと\z{少}{すこ}しだけでいい
  \jisho{}

  
もう\z{少}{すこ}しだけでいいから
  \jisho{}

  
もう\z{少}{すこ}しだけでいい あと\z{少}{すこ}しだけでいい
  \jisho{}

  
もう\z{少}{すこ}しだけ くっついていようか
  \jisho{}

\item
  \z{僕}{ぼく}らタイムフライヤー \z{時}{とき}を\z{駆}{か}け\z{上}{あ}がるクライマー
  \jisho{}

  \z{時}{とき}のかくれんぼ はぐれっこはもういやなんだ
  \jisho{}

\item
  \z{嬉}{うれ}しくて\z{泣}{な}くのは \z{悲}{かな}しくて\z{笑}{わら}うのは
  \jisho{}

  \z{君}{きみ}の\z{心}{こころ}が \z{君}{きみ}を\z{追}{お}い\z{越}{こ}したんだよ
  \jisho{}

\item
  \z{星}{ほし}にまで\z{願}{ねが}って \z{手}{て}にいれたオモチャも
  \jisho{}

  \z{部屋}{へや}の\z{隅}{すみ}っこに\z{今}{いま} \z{転}{ころ}がってる
  \jisho{}

  \z{叶}{かな}えたい\z{夢}{ゆめ}も \z{今日}{きょう}で100\z{個}{こ}できたよ
  \jisho{}

  
たった\z{一}{ひと}つといつか \z{交換}{こうかん}こしよう
  \jisho{}

\item
  
いつもは\z{喋}{しゃべ}らないあの\z{子}{こ}に\z{今日}{きょう}は
  \jisho{}

  \z{放課後}{ほうかご}「また\z{明日}{あした}」と\z{声}{こえ}をかけた
  \jisho{}

  \z{慣}{な}れないこともたまにならいいね
  \jisho{}

  \z{特}{とく}にあなたが \z{隣}{となり}にいたら
  \jisho{}

\item
  
もう\z{少}{すこ}しだけでいい あと\z{少}{すこ}しだけでいい
  \jisho{}

  
もう\z{少}{すこ}しだけでいいから
  \jisho{}

  
もう\z{少}{すこ}しだけでいい あと\z{少}{すこ}しだけでいい
  \jisho{}

  
もう\z{少}{すこ}しだけくっついていようよ
  \jisho{}

\item
  \z{僕}{ぼく}らタイムフライヤー \z{君}{きみ}を\z{知}{し}っていたんだ
  \jisho{}

  \z{僕}{ぼく}が \z{僕}{ぼく}の\z{名前}{なまえ}を \z{覚}{おぼ}えるよりずっと\z{前}{まえ}に
  \jisho{}

\item
  \z{君}{きみ}のいない \z{世界}{せかい}にも \z{何}{なに}かの\z{意味}{いみ}はきっとあって
  \jisho{}

  
でも\z{君}{きみ}のいない \z{世界}{せかい}など \z{夏休}{なつやす}みのない \z{八月}{はちがつ}のよう
  \jisho{}

  \z{君}{きみ}のいない \z{世界}{せかい}など \z{笑}{わら}うことない サンタのよう
  \jisho{}

  \z{君}{きみ}のいない \z{世界}{せかい}など
  \jisho{}

\item
  \z{僕}{ぼく}らタイムフライヤー \z{時}{とき}を\z{駆}{か}け\z{上}{あ}がるクライマー
  \jisho{}

  \z{時}{とき}のかくれんぼ はぐれっこはもういやなんだ
  \jisho{}

\item
  
なんでもないや やっぱりなんでもないや
  \jisho{}

  \z{今}{いま}から\z{行}{い}くよ
  \jisho{}

\item
  \z{僕}{ぼく}らタイムフライヤー \z{時}{とき}を\z{駆}{か}け\z{上}{あ}がるクライマー
  \jisho{}

  \z{時}{とき}のかくれんぼ はぐれっこ はもういいよ
  \jisho{}

\item
  \z{君}{きみ}は\z{派手}{はで}なクライヤー その\z{涙}{なみだ} \z{止}{と}めてみたいな
  \jisho{}

  
だけど \z{君}{きみ}は\z{拒}{こば}んだ \z{零}{こぼ}れるままの\z{涙}{なみだ}を\z{見}{み}てわかった
  \jisho{}

\item
  \z{嬉}{うれ}しくて\z{泣}{な}くのは \z{悲}{かな}しくて \z{笑}{わら}うのは
  \jisho{}

  \z{僕}{ぼく}の\z{心}{こころ}が \z{僕}{ぼく}を\z{追}{お}い\z{越}{こ}したんだよ
  \jisho{}

  
}
\end{document}


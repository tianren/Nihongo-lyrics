% !TeX encoding = UTF-8
% !TeX program = LuaLaTeX

%\documentclass[12pt]{article}
\documentclass[14pt]{ltjsarticle}
% !TeX encoding = UTF-8
% !TeX program = LuaLaTeX

% !TeX encoding = UTF-8
% !TeX program = LuaLaTeX

%\documentclass[12pt]{article}

%\usepackage{luatexja-preset}
%\usepackage{fontspec}
\usepackage[no-math]{luatexja-fontspec}
\usepackage{luatexja-ruby}
\usepackage{color}

\topmargin=-0.95in      %
\evensidemargin=0in     %
\oddsidemargin=0in      %
\textwidth=6.5in        %
\textheight=9.75in       %
\headsep=0.25in

\setmainfont{Noto Sans CJK JP}
\setmainjfont{Noto Sans CJK JP}

\makeatletter
\def\dynscriptsize{\check@mathfonts\fontsize{\sf@size}{\z@}\selectfont}
\makeatother
\def\textunderset#1#2{\leavevmode
  \vtop{\offinterlineskip\halign{%
    \hfil##\hfil\cr\strut#2\cr\noalign{\kern-.3ex}
    \hidewidth\dynscriptsize\strut#1\hidewidth\cr}}}
\newcommand{\righttext}[1]{\ifx&#1&\else
  \hspace{1em} \unskip
  \nobreak \hfil \penalty1000 \hfilneg \indent
  \strut\hfill \mbox{\def\z{\zhuyin}\small #1}
  \fi}

\newcommand{\zhuyin}[2]{\ruby{#1}{#2}}
\newcommand{\kanji}[2]{\textunderset{#1}{\textcolor{blue}{#2}}}
\def\jisho{\righttext}
\def\trans{\righttext}

\newcommand{\lyrics}[1]{\begingroup
\def\z{\zhuyin}
\list{}{\leftmargin0em}
\setlength\itemsep{1em}
  #1
\endlist
\endgroup}

\makeatletter
\newcommand{\itemmark}[2]{%
\item[#1.]
\def\@currentlabel{#1}\label{par:#1}
  \expandafter\def\csname firstline@#1\endcsname{#2}%
  #2
}
\newcommand{\itemrepeat}[1]{
  \item[\ref{par:#1}.]
  \csname firstline@#1\endcsname
}
\makeatother


\renewcommand{\jisho}[1]{}

\renewcommand{\lyrics}[1]{\begingroup
\def\z{\zhuyin}
\list{}{\leftmargin0em}
\setlength\itemsep{1em}
  #1
\endlist
\endgroup
\clearpage
\begingroup
\def\z{\kanji}
\list{}{\leftmargin0em}
\setlength\itemsep{1em}
  #1
\endlist
\endgroup}

\begin{document}

\lyrics{
\item
  \textbf{二言目 \hfill %
    「偽物語」のOP}

\item
  \z{今}{いま}だって \z{君}{きみ}のとなりで
  \jisho{}

  いつだって \z{不安}{ふあん}になるよ
  \jisho{}

\item
  \z{窓}{まど}の\z{外}{そと} \z{見}{み}ているフリで
  \jisho{}

  ガラスに\z{映}{うつ}る \z{君}{きみ}を\z{見}{み}ていた
  \jisho{}

\item
  \z{君}{きみ}は \z{私}{わたし}の どんなとこを \z{好}{す}きになってくれたの? とか
  \jisho{}

  \z{今日}{きょう}も そんな \z{私}{わたし}で ちゃんと いられてるか いられてないか
  \jisho{}

  \z{行}{い}ったり \z{来}{き}たり もぅ ぐるぐるぐるぐる
  \jisho{}

\item
  その \z{一言}{ひとこと}でね
  \jisho{}

  ほら \z{全部}{ぜんぶ} \z{全部}{ぜんぶ} \z{忘}{わす}れちゃって
  \jisho{}

  その \z{二言目}{ふたことめ}で
  \jisho{}

  また もっと もっと よくばりになる
  \jisho{}

  \z{君}{きみ}の\z{前}{まえ}だと \z{全部}{ぜんぶ} ほどかれてくの どこまでも
  \jisho{}

\item
  \z{聴}{き}こえない\z{位}{くらい}の\z{声}{こえ}で ささやいた
  \jisho{}

  \z{君}{きみ}が\z{好}{す}きだよ
  \jisho{}

\item
  \z{君}{きみ}は ねぇ \z{気付}{きづ}いているの?
  \jisho{}

  \z{息}{いき}をひそめて \z{答}{こたえ}を\z{待}{ま}った
  \jisho{}

\item
  \z{誰}{だれ}かの \z{不幸}{ふしあわ}せの\z{上}{うえ}に \z{築}{きず}く \z{幸}{しあわ}せの\z{意味}{いみ}とか
  \jisho{}

  \z{全}{すべ}て \z{自分}{じぶん}で \z{決}{き}めたくせに \z{正}{ただ}しかったのか \z{間違}{まちが}ってたのか
  \jisho{}

  \z{行}{い}ったり \z{来}{き}たり もぅ ぐるぐるぐるぐる
  \jisho{}

\item
  その \z{一言}{ひとこと}でね
  \jisho{}

  ほら \z{全部}{ぜんぶ} \z{全部}{ぜんぶ} \z{忘}{わす}れちゃって
  \jisho{}

  その \z{二言目}{ふたことめ}で
  \jisho{}

  また もっと もっと よくばりになる
  \jisho{}

  \z{君}{きみ}の\z{前}{まえ}だと \z{全部}{ぜんぶ} ほどかれてくの どこまでも
  \jisho{}

\item
  \z{言葉}{ことば}で\z{身体}{からだ}に \z{鍵}{かぎ}をかけたって
  \jisho{}

  \z{心}{こころ}の\z{中}{なか}まで \z{縛}{しば}れないけど..
  \jisho{}

\item
  その \z{一言}{ひとこと}\z{目}{め}が
  \jisho{}

  ねぇ もしも もしも \z{君}{きみ}だったら
  \jisho{}

  \z{続}{つづ}く \z{二言目}{ふたことめ}
  \jisho{}

  そぅ いつも \z{私}{わたし}だといいな
  \jisho{}

\item
  その \z{一言}{ひとこと}でね
  \jisho{}

  ほら \z{全部}{ぜんぶ} \z{全部}{ぜんぶ} \z{忘}{わす}れちゃって
  \jisho{}

  その \z{二言目}{ふたことめ}で
  \jisho{}

  また もっと もっと よくばりになる
  \jisho{}

  \z{君}{きみ}の\z{前}{まえ}だと だめだ.. ほどかれてくの どこまでも
  \jisho{}


}
\end{document}

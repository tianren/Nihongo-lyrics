% !TeX encoding = UTF-8
% !TeX program = LuaLaTeX

%\documentclass[12pt]{article}
\documentclass[14pt]{ltjsarticle}
../macros/dual-lyrics-luamacros.tex
\begin{document}

\lyrics{
\item
  \textbf{たからもの \hfill %
    「\z{夏目}{なつめ}\z{友人}{ゆうじん}\z{帳}{ちょう} \z{肆}{し}」のED}

\item
  \z{微笑}{ほほえ}んだうしろ\z{姿}{すがた}に
  \jisho{}

  \z{泣}{な}きそうな\z{顔}{かお}を \z{隠}{かく}してた
  \jisho{}

  やさしさで\z{胸}{むね}がいたくて
  \jisho{}

  こんなに\z{日々}{ひび}がいとおしくて
  \jisho{}

\item
  \z{風}{かぜ}の\z{音}{おと}に \z{夕闇}{ゆうやみ}に \z{懐}{なつ}かしい\z{君}{きみ}を\z{思}{おも}い\z{出}{だ}す
  \jisho{}

  いつまでも\z{一緒}{いっしょ}だよ、と \z{叶}{かな}わぬことくり\z{返}{かえ}し
  \jisho{}

\item
  ぬくもりはこの\z{手}{て}に
  \jisho{}

  あざやかなまま \z{生}{い}きている
  \jisho{}

  \z{忘}{わす}れたくないもの
  \jisho{}

  \z{受}{う}け\z{取}{と}った\z{愛}{あい}を \z{未来}{みらい}にかえながら
  \jisho{}

\item
  あと\z{何}{なに}を\z{話}{はな}せただろう
  \jisho{}

  はなれてしまう その\z{前}{まえ}に
  \jisho{}

  \z{淋}{さび}しさを\z{感}{かん}じることは
  \jisho{}

  \z{孤独}{こどく}とどこか\z{違}{ちが}っていて
  \jisho{}

\item
  ありがとう うれしいよ \z{大切}{たいせつ}な\z{時間}{じかん}をくれたね
  \jisho{}

  \z{永遠}{えいえん}をつなぐように \z{深}{ふか}い\z{場所}{ばしょ}でふれあえた
  \jisho{}

\item
  しあわせの\z{余韻}{よいん}が
  \jisho{}

  そっと\z{背中}{せなか}を \z{押}{お}している
  \jisho{}

  \z{旅立}{たびだ}ちの\z{夜明}{よあ}けを
  \jisho{}

  \z{照}{て}らしてたのは \z{二度}{にど}と\z{会}{あ}えない\z{日々}{ひび}
  \jisho{}

\item
  さよならのかわりに \z{抱}{だ}きしめていくんだ
  \jisho{}

  わたしをつくる ひとつひとつを
  \jisho{}

  ずっと…
  \jisho{}

\item
  ぬくもりはこの\z{手}{て}に
  \jisho{}

  あざやかなまま \z{生}{い}きている
  \jisho{}

  \z{忘}{わす}れたりしないよ
  \jisho{}

  \z{受}{う}け\z{取}{と}った\z{愛}{あい}の \z{輝}{かがや}きと\z{歩}{ある}こう
  \jisho{}

  
}
\end{document}


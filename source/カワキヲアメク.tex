% !TeX encoding = UTF-8
% !TeX program = LuaLaTeX

%\documentclass[12pt]{article}
\documentclass[14pt]{ltjsarticle}
% !TeX encoding = UTF-8
% !TeX program = LuaLaTeX

% !TeX encoding = UTF-8
% !TeX program = LuaLaTeX

%\documentclass[12pt]{article}

%\usepackage{luatexja-preset}
%\usepackage{fontspec}
\usepackage[no-math]{luatexja-fontspec}
\usepackage{luatexja-ruby}
\usepackage{color}

\topmargin=-0.95in      %
\evensidemargin=0in     %
\oddsidemargin=0in      %
\textwidth=6.5in        %
\textheight=9.75in       %
\headsep=0.25in

\setmainfont{Noto Sans CJK JP}
\setmainjfont{Noto Sans CJK JP}

\makeatletter
\def\dynscriptsize{\check@mathfonts\fontsize{\sf@size}{\z@}\selectfont}
\makeatother
\def\textunderset#1#2{\leavevmode
  \vtop{\offinterlineskip\halign{%
    \hfil##\hfil\cr\strut#2\cr\noalign{\kern-.3ex}
    \hidewidth\dynscriptsize\strut#1\hidewidth\cr}}}
\newcommand{\righttext}[1]{\ifx&#1&\else
  \hspace{1em} \unskip
  \nobreak \hfil \penalty1000 \hfilneg \indent
  \strut\hfill \mbox{\def\z{\zhuyin}\small #1}
  \fi}

\newcommand{\zhuyin}[2]{\ruby{#1}{#2}}
\newcommand{\kanji}[2]{\textunderset{#1}{\textcolor{blue}{#2}}}
\def\jisho{\righttext}
\def\trans{\righttext}

\newcommand{\lyrics}[1]{\begingroup
\def\z{\zhuyin}
\list{}{\leftmargin0em}
\setlength\itemsep{1em}
  #1
\endlist
\endgroup}

\makeatletter
\newcommand{\itemmark}[2]{%
\item[#1.]
\def\@currentlabel{#1}\label{par:#1}
  \expandafter\def\csname firstline@#1\endcsname{#2}%
  #2
}
\newcommand{\itemrepeat}[1]{
  \item[\ref{par:#1}.]
  \csname firstline@#1\endcsname
}
\makeatother


\renewcommand{\jisho}[1]{}

\renewcommand{\lyrics}[1]{\begingroup
\def\z{\zhuyin}
\list{}{\leftmargin0em}
\setlength\itemsep{1em}
  #1
\endlist
\endgroup
\clearpage
\begingroup
\def\z{\kanji}
\list{}{\leftmargin0em}
\setlength\itemsep{1em}
  #1
\endlist
\endgroup}

\begin{document}

\lyrics{
\item
  \textbf{カワキヲアメク \hfill %
    「ドメスティックな彼女」のOP}

\item
  \z{未熟}{みじゅく} \z{無}{む}ジョウ されど \z{美}{うつく}しくあれ
  \jisho{}

\item
  \z{No Destiny}{ノーディスティニー} ふさわしく\z{無}{な}い
  \jisho{}

  
こんなんじゃきっと\z{物足}{ものた}りない
  \jisho{}

  
くらい\z{語}{かた}っとけばうまくいく
  \jisho{}

  \z{物}{もの}、\z{金}{かね}、\z{愛}{あい}、\z{言}{こと}、もう\z{自己}{じこ}\z{顕示}{けんじ}\z{飽}{あ}きた
  \jisho{}

\item
  \z{既視感}{デジャヴ} \z{何}{なに}がそんな\z{不満}{ふまん}なんだ?
  \jisho{}

  \z{散々}{さんざん}ワガママ\z{語}{かた}っといて これ\z{以上}{いじょう}\z{他}{ほか}に\z{何}{なに}がいる?
  \jisho{}

  
そんなところも\z{割}{わり}と\z{嫌}{きら}いじゃ\z{無}{な}い
  \jisho{}

\item
  
もう「\z{聞}{き}き\z{飽}{あ}きたんだよ、そのセリフ。」
  \jisho{}

  \z{中途半端}{ちゅうとはんぱ}だけは\z{嫌}{いや}
  \jisho{}

\item
  
もういい
  \jisho{}

  
ああしてこうして\z{言}{い}ってたって
  \jisho{}

  \z{愛}{あい}して どうして? \z{言}{い}われたって
  \jisho{}

  \z{遊}{あそ}びだけなら\z{簡単}{かんたん}で \z{真剣}{しんけん}\z{交渉}{こうしょう}\z{無茶苦茶}{むちゃくちゃ}で
  \jisho{}

  \z{思}{おも}いもしない\z{軽い}{おもい}\z{言葉}{ことば}
  \jisho{}

  \z{何度}{なんど}\z{使}{つか}い\z{古}{ふる}すのか?
  \jisho{}

  
どうせ
  \jisho{}

  \z{期待}{きたい}してたんだ\z{出来}{でき}レースでも
  \jisho{}

  \z{引用}{いんよう}だらけのフレーズも
  \jisho{}

  \z{踵}{かかと}\z{持}{も}ち\z{上}{あ}がる\z{言葉}{ことば}タブーにして
  \jisho{}

  \z{空気}{くうき}を\z{読}{よ}んだ\z{雨}{あめ}\z{降}{ふ}らないでよ
  \jisho{}

\item
  
まどろっこしい\z{話}{はなし}は\z{嫌}{いや}
  \jisho{}

  \z{必要}{ひつよう}\z{最低限}{さいていげん}でいい 2\z{文字}{もじ}\z{以内}{いない}でどうぞ
  \jisho{}

\item
  \z{紅}{くれない}の\z{蝶}{ちょう}は\z{何}{なん}のメールも\z{送}{おく}らない
  \jisho{}

  \z{脆}{もろ}い\z{扇子}{せんす}\z{広}{ひろ}げる その\z{方}{ほう}が\z{魅力的}{みりょくてき}でしょう
  \jisho{}

\item
  \z{迷}{めい}で
  \jisho{}

  \z{応}{こた}えられないなら ほっといてくれ
  \jisho{}

  \z{迷}{まよ}えるくらいなら \z{去}{さ}っといてくれ
  \jisho{}

  \z{肝心}{かんじん}なとこは\z{筒抜}{つつぬ}けで \z{安心}{あんしん}だけはさせられるような
  \jisho{}

  \z{甘}{あま}いあめが\z{降}{ふ}れば
  \jisho{}

  \z{傘}{かさ}もさしたくなるだろう?
  \jisho{}

  
このまま
  \jisho{}

  \z{期待}{きたい}したままでよかった \z{目}{め}を\z{瞑}{つぶ}った
  \jisho{}

  \z{変}{か}えたかった \z{大人}{おとな}ぶった
  \jisho{}

  \z{無}{な}くした \z{巻}{ま}き\z{戻}{もど}せなかった
  \jisho{}

  \z{今}{いま}\z{雨}{あめ}、\z{止}{や}まないで
  \jisho{}

\item
  
コピー、ペースト、デリート その\z{繰}{く}り\z{返}{かえ}し
  \jisho{}

  \z{吸}{す}って、\z{吐}{は}いた
  \jisho{}

  
だから
  \jisho{}

  
それでもいいからさ \z{此処}{ここ}いたいよ
  \jisho{}

\item
  
もういい
  \jisho{}

  
ああしてこうして\z{言}{い}ってたって
  \jisho{}

  \z{愛}{あい}して どうして? \z{言}{い}われたって
  \jisho{}

  \z{遊}{あそ}びだけなら\z{簡単}{かんたん}で \z{真剣}{しんけん}\z{交渉}{こうしょう}\z{支離滅裂}{しりめつれつ}で
  \jisho{}

  \z{思}{おも}いもしない\z{重}{おも}い\z{真実}{うそ}は
  \jisho{}

  
タブーにしなくちゃな?
  \jisho{}

  
きっと
  \jisho{}

  \z{期待}{きたい}してたんだ\z{出来}{でき}レースでも
  \jisho{}

  \z{公式}{こうしき}\z{通}{どお}りのフレーズも
  \jisho{}

  \z{踵}{かかと}\z{上}{あ}がる\z{癖}{くせ}もう\z{終}{お}わりにして
  \jisho{}

  \z{空気}{くうき}を\z{読}{よ}んだ\z{空}{そら}\z{晴}{は}れないでよ
  \jisho{}

\item
  \z{今日}{きょう}も、\z{雨}{あめ}。
  \jisho{}

  \z{傘}{かさ}を\z{閉}{と}じて \z{濡}{ぬ}れて\z{帰}{かえ}ろうよ
  \jisho{}

  
}
\end{document}


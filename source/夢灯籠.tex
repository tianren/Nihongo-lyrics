% !TeX encoding = UTF-8
% !TeX program = LuaLaTeX

%\documentclass[12pt]{article}
\documentclass[14pt]{ltjsarticle}
% !TeX encoding = UTF-8
% !TeX program = LuaLaTeX

% !TeX encoding = UTF-8
% !TeX program = LuaLaTeX

%\documentclass[12pt]{article}

%\usepackage{luatexja-preset}
%\usepackage{fontspec}
\usepackage[no-math]{luatexja-fontspec}
\usepackage{luatexja-ruby}
\usepackage{color}

\topmargin=-0.95in      %
\evensidemargin=0in     %
\oddsidemargin=0in      %
\textwidth=6.5in        %
\textheight=9.75in       %
\headsep=0.25in

\setmainfont{Noto Sans CJK JP}
\setmainjfont{Noto Sans CJK JP}

\makeatletter
\def\dynscriptsize{\check@mathfonts\fontsize{\sf@size}{\z@}\selectfont}
\makeatother
\def\textunderset#1#2{\leavevmode
  \vtop{\offinterlineskip\halign{%
    \hfil##\hfil\cr\strut#2\cr\noalign{\kern-.3ex}
    \hidewidth\dynscriptsize\strut#1\hidewidth\cr}}}
\newcommand{\righttext}[1]{\ifx&#1&\else
  \hspace{1em} \unskip
  \nobreak \hfil \penalty1000 \hfilneg \indent
  \strut\hfill \mbox{\def\z{\zhuyin}\small #1}
  \fi}

\newcommand{\zhuyin}[2]{\ruby{#1}{#2}}
\newcommand{\kanji}[2]{\textunderset{#1}{\textcolor{blue}{#2}}}
\def\jisho{\righttext}
\def\trans{\righttext}

\newcommand{\lyrics}[1]{\begingroup
\def\z{\zhuyin}
\list{}{\leftmargin0em}
\setlength\itemsep{1em}
  #1
\endlist
\endgroup}

\makeatletter
\newcommand{\itemmark}[2]{%
\item[#1.]
\def\@currentlabel{#1}\label{par:#1}
  \expandafter\def\csname firstline@#1\endcsname{#2}%
  #2
}
\newcommand{\itemrepeat}[1]{
  \item[\ref{par:#1}.]
  \csname firstline@#1\endcsname
}
\makeatother


\renewcommand{\jisho}[1]{}

\renewcommand{\lyrics}[1]{\begingroup
\def\z{\zhuyin}
\list{}{\leftmargin0em}
\setlength\itemsep{1em}
  #1
\endlist
\endgroup
\clearpage
\begingroup
\def\z{\kanji}
\list{}{\leftmargin0em}
\setlength\itemsep{1em}
  #1
\endlist
\endgroup}

\begin{document}

\lyrics{
\item
  \textbf{\zhuyin{夢}{ゆめ}\zhuyin{灯籠}{とうろう} \hfill %
    「君の名は。」の\zhuyin{主}{しゅ}\zhuyin{題}{だい}\zhuyin{歌}{か}}

\item
  あぁ このまま\z{僕}{ぼく}たちの\z{声}{こえ}が
  \jisho{}

  \z{世界}{せかい}の\z{端}{はし}っこまで\z{消}{き}えることなく
  \jisho{}

  \z{届}{とど}いたりしたらいいのにな
  \jisho{}

  そしたらねぇ \z{二人}{ふたり}で
  \jisho{}

  どんな\z{言葉}{ことば}を\z{放}{はな}とう
  \jisho{}

  \z{消}{き}えることない\z{約束}{やくそく}を
  \jisho{}

  \z{二人}{ふたり}で「せーの」で  \z{言}{い}おう
  \jisho{}

\item
  あぁ「\z{願}{ねが}ったらなにがしかが\z{叶}{かな}う」
  \jisho{}

  その\z{言葉}{ことば}の\z{眼}{め}を
  \jisho{}

  もう\z{見}{み}れなくなったのは
  \jisho{}

  \z{一体}{いったい}いつからだろうか
  \jisho{}

  なにゆえだろうか
  \jisho{}

  あぁ \z{雨}{あめ}の\z{止}{や}むまさにその\z{切}{き}れ\z{間}{ま}と
  \jisho{}

  \z{虹}{にじ}の\z{出発点}{しゅっぱつてん} \z{終点}{しゅうてん}と
  \jisho{}

  この\z{命果}{いのちは}てる\z{場所}{ばしょ}に\z{何}{なに}かがあるって
  \jisho{}

  いつも\z{言}{い}い\z{張}{は}っていた
  \jisho{}

\item
  いつか\z{行}{い}こう \z{全生命}{ぜんせいめい}も\z{未到}{みとう}
  \jisho{}

  \z{未開拓}{みかいたく}の
  \jisho{}

  \z{感情}{かんじょう}にハイタッチして
  \jisho{}

  \z{時間}{じかん}にキスを
  \jisho{}

  5\z{次元}{じげん}にからかわれて
  \jisho{}

  それでも\z{君}{きみ}をみるよ
  \jisho{}

  また「はじめまして」の\z{合図}{あいず}を
  \jisho{}

  \z{決}{き}めよう
  \jisho{}

  \z{君}{きみ}の\z{名}{な}を \z{今追}{いまお}いかけるよ
  \jisho{}

}
\end{document}

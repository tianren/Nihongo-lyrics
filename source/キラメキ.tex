% !TeX encoding = UTF-8
% !TeX program = LuaLaTeX

%\documentclass[12pt]{article}
\documentclass[14pt]{ltjsarticle}
../macros/dual-lyrics-luamacros.tex
\begin{document}

\lyrics{
\item
  \textbf{キラメキ \hfill %
    「四月は君の嘘」のED}

\item
  \z{落}{お}ち\z{込}{こ}んでた\z{時}{とき}も \z{気}{き}がつけば\z{笑}{わら}ってる
  \jisho{}

  \z{二人}{ふたり}なら \z{世界}{せかい}は\z{息}{いき}を\z{吹}{ふ}き\z{返}{かえ}した
  \jisho{}

\item
  
いつもの\z{帰}{かえ}り\z{道}{みち} \z{足音}{あしおと}\z{刻}{きざ}むリズム
  \jisho{}

  \z{雨上}{あめあ}がり \z{街}{まち}を\z{抜}{ぬ}けてゆく\z{風}{かぜ}の\z{優}{やさ}しい\z{匂}{にお}い
  \jisho{}

\item
  \z{同}{おな}じ\z{時間}{じかん}を\z{分}{わ}け\z{合}{あ}いながら \z{二人}{ふたり}で\z{過}{す}ごせた\z{奇跡}{きせき}を
  \jisho{}

  
これから\z{先}{さき}も\z{繋}{つな}げたいんだ ちゃんと\z{目}{め}を\z{見}{み}て\z{伝}{つた}えたい
  \jisho{}

\item
  
つないでいたい\z{手}{て}は \z{君}{きみ}のものだったよ
  \jisho{}

  \z{握}{にぎ}り\z{方}{かた}で\z{何}{なに}もかもを\z{伝}{つた}え\z{合}{あ}える その\z{手}{て}だった
  \jisho{}

  \z{他}{ほか}の\z{誰}{だれ}でもない \z{君}{きみ}じゃなきゃだめだよ
  \jisho{}

  
いつまでもそばにいたいと\z{思}{おも}えた
  \jisho{}

\item
  \z{振}{ふ}り\z{返}{かえ}ってみても いないのは\z{分}{わ}かってる
  \jisho{}

  
なのにまた \z{名前}{なまえ}\z{呼}{よ}ばれた\z{気}{き}がして \z{見渡}{みわた}してみる
  \jisho{}

\item
  \z{角}{かど}を\z{曲}{ま}がれば \z{歩幅}{ほはば}\z{合}{あ}わせた あの\z{頃}{ころ}に\z{戻}{もど}れるような
  \jisho{}

  \z{桜}{さくら}のアーチ \z{今}{いま}はその\z{葉}{は}を オレンジに\z{染}{そ}めてるけど
  \jisho{}

\item
  \z{咲}{さ}かせたい\z{笑顔}{えがお}は \z{君}{きみ}のものだったよ
  \jisho{}

  \z{街}{まち}\z{彩}{いろど}る\z{木々}{きぎ}のように \z{綺麗}{きれい}な\z{赤}{あか}い その\z{頬}{ほお}だった
  \jisho{}

  \z{思}{おも}い\z{出}{で}が\z{舞}{ま}い\z{散}{ち}る こみ\z{上}{あ}げる\z{想}{おも}いを
  \jisho{}

  
どこまでも\z{遠}{とお}い\z{空}{そら}へと \z{飛}{と}ばした
  \jisho{}

\item
  \z{聞}{き}いていたい\z{声}{こえ}は \z{君}{きみ}のものだったよ
  \jisho{}

  \z{耳}{みみ}を\z{伝}{つた}い\z{体中}{からだじゅう}を\z{包}{つつ}むような その\z{声}{こえ}だった
  \jisho{}

  \z{出会}{であ}いから\z{全}{すべ}てが かけがえのない\z{日々}{ひび}
  \jisho{}

  
いつまでもこの\z{胸}{むね}にあるよ ありがとう
  \jisho{}

  
}
\end{document}


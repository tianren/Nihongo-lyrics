% !TeX encoding = UTF-8
% !TeX program = LuaLaTeX

%\documentclass[12pt]{article}
\documentclass[14pt]{extreport}
../macros/dual-lyrics-luamacros.tex
\begin{document}

\lyrics{
\item
  \textbf{君が笑む夕暮れ \hfill 「東京レイヴンズ」のED1}

\item
  もう、この\z{季}{き}\z{節}{せつ}も\z{終}{お}わりだね…と \z{君}{きみ}が\z{不}{ふ}\z{意}{い}につぶやく
  \jisho{\z{不}{ふ}\z{意}{い}に unexpectedly, \z{呟}{つぶや}く to mutter}

  ああ…\z{何}{なに}\z{気}{げ}なくて \z{優}{やさ}しすぎるから \z{奥}{おく}\z{歯}{ば} かんだ
  \jisho{\z{何}{なに}\z{気}{げ}なく unintentionally, \z{過}{す}ぎる [ru] to be too much, \z{噛}{か}む to bite}

\item
  ねえ、\z{繋}{つな}がりとか\z{運}{うん}\z{命}{めい}とか そんなものを \z{捨}{す}てたら
  \jisho{\z{繋}{つな}がり connection, \z{捨}{す}てる [ru] to discard}

  もっと\z{自}{じ}\z{由}{ゆう}に この\z{空}{そら}\z{飛}{と}べるのかな?
  \jisho{もっと more, \z{飛}{と}ぶ to fly, かな I wonder}

\item
  \z{教}{おし}えてよ…
  \jisho{}

\item
  いつか\z{交}{か}わした\z{約}{やく}\z{束}{そく} \z{目}{め}\z{指}{ざ}す\z{場}{ば}\z{所}{しょ}は\z{高}{たか}く
  \jisho{\z{何}{い}\z{時}{つ}か someday, \z{交}{か}わす to exchange, \z{約}{やく}\z{束}{そく} promise,
    \z{目}{め}\z{指}{ざ}す to aim at}

  \z{日}{にち}\z{常}{じょう}の\z{騒}{さわ}がしさと\z{戯}{たわむ}れるけど
  \jisho{\z{騒}{さわ}がしい noisy, \z{戯}{たわむ}れる [ru] to be playful}

  \z{君}{きみ}のために\z{強}{つよ}くなる \z{今}{いま}\z{気}{き}\z{付}{づ}いた\z{欠}{かけ}\z{片}{ら}
  \jisho{ために for the sake of, \z{強}{つよ}い, \z{成}{な}る to become,
    \z{気}{き}\z{付}{づ}く to notice, \z{欠}{かけ}\z{片}{ら} fragment}

  \z{手}{て}\z{放}{ばな}してしまわないように
  \jisho{(\z{手}{て})\z{放}{ばな}す to let go, しまう to finish, \z{様}{よう}に in order to}

  だけど\z{今}{いま}は こっち\z{見}{み}ないで 
  \jisho{だけど however, こっち this direction; me}

  \z{濡}{ぬ}れた\z{頰}{ほお} \z{夕}{ゆう}\z{日}{ひ}が\z{乾}{かわ}かすまで
  \jisho{\z{濡}{ぬ}れる [ru] to get wet, \z{乾}{かわ}かす to dry}

\item
  もう\z{放}{ほ}っといて…と\z{強}{つよ}がるのは \z{君}{きみ}の\z{悪}{わる}い\z{癖}{くせ}だね
  \jisho{\z{放}{ほ}つ to free, \z{放}{ほ}っとく to leave alone, \z{強}{つよ}がる to pretend to be tough}

  そう\z{言}{い}って\z{肩}{かた}を \z{小}{こ}\z{突}{づ}いた\z{笑}{え}\z{顔}{がお}に \z{瞳}{ひとみ}\z{伏}{ふ}せた
  \jisho{\z{小}{こ}\z{突}{づ}く to push, \z{伏}{ふ}せる [ru]}

  ねえ、\z{僕}{ぼく}が\z{書}{か}いたあらすじなど \z{子}{こ}\z{供}{ども}\z{過}{す}ぎて ちっぽけで
  \jisho{}

  \z{真}{しん}\z{実}{じつ}を\z{知}{し}れば \z{脆}{もろ}く\z{崩}{くず}れるのだろう
  \jisho{}

\item
  それでもさ…
  \jisho{}

\item
  \z{暮}{く}れる\z{空}{そら}に\z{夢}{ゆめ}\z{見}{み}てる \z{思}{おも}う\z{時}{とき}は\z{長}{なが}く
  \jisho{}

  \z{日}{にち}\z{常}{じょう}のすれ\z{違}{ちが}いで\z{千}{ち}\z{切}{ぎ}れそうでも
  \jisho{}

  \z{君}{きみ}はもっと\z{強}{つよ}くなる-\z{風}{かぜ}に\z{乗}{の}る\z{言}{こと}\z{霊}{だま}
  \jisho{}

  \z{今}{いま}はただ \z{追}{お}いかけてるんだ
  \jisho{}

  だけど\z{今日}{きょう}は \z{少}{すこ}し\z{寒}{さむ}くて \z{繋}{つな}いだ\z{君}{きみ}の\z{手}{て} \z{解}{ほど}けないよ…
  \jisho{}

\item
  \z{今}{いま}はまだ
  \jisho{}

\item
  \z{知}{し}らない\z{道}{みち}の\z{途}{と}\z{中}{ちゅう}で \z{出}{で}\z{会}{あ}いを\z{繰}{く}り\z{返}{かえ}す
  \jisho{}

  \z{日}{にち}\z{常}{じょう}の\z{喧}{けん}\z{噪}{そう}さえ \z{愛}{いと}おしいけど
  \jisho{}

  \z{君}{きみ}の\z{声}{こえ}が\z{遠}{とお}くなる その\z{瞬}{しゅん}\z{間}{かん} \z{僕}{ぼく}が\z{心}{こころ}から\z{笑}{わら}ってますように…
  \jisho{}

\item
  \z{小}{ちい}さな\z{願}{ねが}い \z{強}{つよ}く\z{結}{むす}んで
  \jisho{}

  いつま\z{通}{どお}り\z{横}{よこ}\z{顔}{がお}を \z{見}{み}ていた…
  \jisho{}

\item
  \z{振}{ふ}り\z{返}{かえ}る\z{君}{きみ}が\z{今}{いま}…… \z{笑}{わら}った
  \jisho{}
}

\end{document}

% !TeX encoding = UTF-8
% !TeX program = LuaLaTeX

%\documentclass[12pt]{article}
\documentclass[14pt]{ltjsarticle}
../macros/dual-lyrics-luamacros.tex
\begin{document}

\lyrics{
\item
  \textbf{リアルワールド \hfill %
    「\z{人類}{じんるい}は\z{衰退}{すいたい}しました」のOP}

\item
  \z{目}{め}が\z{覚}{さ}めたならきみが\z{笑}{わら}ってそんな\z{世界}{せかい}が\z{続}{つづ}くと\z{思}{おも}ってた
  \jisho{}

  \z{当}{あ}たり\z{前}{まえ}には\z{少}{すこ}し\z{足}{た}りない\z{歪}{ゆが}んだ\z{視界}{しかい}から\z{見}{み}てた\z{青}{あお}い\z{夜}{よる}
  \jisho{}

\item
  \z{何度目}{なんどめ}の\z{朝}{あさ}で\z{打}{う}ち\z{明}{あ}けた\z{恋}{こい}のように
  \jisho{}

  \z{少}{すこ}し\z{酸}{す}っぱいままで\z{出掛}{でか}けたら
  \jisho{}

\item
  \z{近付}{ちかづ}いてくぼくらデリケート \z{淡}{あわ}い\z{夢}{ゆめ}を\z{見}{み}せてあげよう
  \jisho{}

  たまにはいいことあるかも
  \jisho{}

  ご\z{褒美}{ほうび}にはチョコレート \z{甘}{あま}い\z{夢}{ゆめ}を\z{見}{み}れたら
  \jisho{}

  それがすべてだなんて\z{笑}{わら}ってみよう
  \jisho{}

\item
  \z{曖昧}{あいまい}だって\z{大体}{だいたい}だって\z{続}{つづ}く\z{気}{き}がして\z{夜空}{よぞら}を\z{仰}{あお}いだ
  \jisho{}

  なんとなくから\z{見}{み}えた\z{景色}{けしき}が\z{新}{あたら}しい\z{世界}{せかい}へほらね\z{導}{みちび}くよ
  \jisho{}

\item
  \z{回}{まわ}り\z{続}{つづ}けるこの\z{星}{ほし}はだれのもの?
  \jisho{}

  \z{難}{むずか}しいハナシなら\z{食後}{しょくご}にして
  \jisho{}

\item
  \z{近付}{ちかづ}いてくぼくらデリケート \z{淡}{あわ}い\z{夢}{ゆめ}を\z{見}{み}せてあげよう
  \jisho{}

  たまにはいいことあるかも
  \jisho{}

  ご\z{褒美}{ほうび}にはチョコレート \z{甘}{あま}い\z{夢}{ゆめ}を\z{見}{み}れたら
  \jisho{}

  それがすべてだなんて\z{笑}{わら}い\z{飛}{と}ばそう
  \jisho{}

\item
  あの\z{丘}{おか}まで\z{進}{すす}めストレート \z{見}{み}えない\z{音}{おと}に\z{耳傾}{みみかたむ}け
  \jisho{}

  \z{聴}{き}こえた?\z{手招}{てまね}きする\z{声}{こえ}
  \jisho{}

  いくつかのバリケード\z{壊}{こわ}せ \z{知}{し}れば\z{知}{し}るほど
  \jisho{}

  わからなくもなるくらい\z{不思議}{ふしぎ}な\z{世界}{せかい}
  \jisho{}

  
}
\end{document}


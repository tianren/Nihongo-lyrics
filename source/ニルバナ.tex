% !TeX encoding = UTF-8
% !TeX program = LuaLaTeX

%\documentclass[12pt]{article}
\documentclass[14pt]{extreport}
../macros/dual-lyrics-luamacros.tex
\begin{document}

\lyrics{
\item
  \textbf{ニルバナ \hfill %
    「ノラガミ」のED}

\item
  やがて\z{巡}{めぐ}り\z{巡}{めぐ}る\z{季節}{きせつ}に
  \jisho{}

  \z{僕}{ぼく}らは\z{息}{いき}をする
  \jisho{}

  \z{思}{おも}い\z{出}{で}になるその\z{時}{とき}まで
  \jisho{}

  ずっと\z{忘}{わす}れないで
  \jisho{}

\item
  \z{一人}{ひとり}ぼっち\z{膝}{ひざ}を\z{抱}{かか}えて
  \jisho{}

  \z{見上}{みあ}げたんだ あの\z{日}{ひ}
  \jisho{}

  \z{思}{おも}ってたより\z{晴}{は}れた\z{空}{そら}と
  \jisho{}

  あなたがそこにいた
  \jisho{}

\item
  \z{見}{み}えてるもの\z{全}{すべ}て \z{胸}{むね}に\z{焼}{や}き\z{付}{つ}けたんだ
  \jisho{}

  いつか\z{来}{く}るさよならの\z{時}{とき}のため
  \jisho{}

  だけど\z{今}{いま}は\z{気}{き}づかぬふりをして
  \jisho{}

  \z{僕}{ぼく}は\z{笑}{わら}う あなたと\z{今}{いま}
  \jisho{}

\item
  \z{悲}{かな}しみ \z{喜}{よろこ}び \z{心臓}{しんぞう}の\z{鼓動}{こどう}
  \jisho{}

  \z{伝}{つた}って\z{動}{うご}かすんだ \z{僕}{ぼく}という\z{命}{いのち}
  \jisho{}

  \z{想}{おも}いや\z{感情}{かんじょう} \z{掛}{か}け\z{値}{ね}なしの\z{愛}{あい}を
  \jisho{}

  あなたがくれたから
  \jisho{}

  \z{進}{すす}むよ \z{見}{み}ててくれる?
  \jisho{}

\item
  \z{真夜中}{まよなか}の\z{雨}{あめ}が\z{降}{ふ}り\z{止}{や}めば
  \jisho{}

  \z{僕}{ぼく}はきっと\z{遠}{とお}く
  \jisho{}

  \z{心配}{しんぱい}しないで \z{同}{おな}じ\z{空}{そら}の
  \jisho{}

  \z{下}{した}に\z{僕}{ぼく}はいるよ
  \jisho{}

\item
  \z{見}{み}えてるもの\z{全}{すべ}て \z{守}{まも}ろうとするほどに
  \jisho{}

  あなたは\z{優}{やさ}しさで\z{傷}{きず}つくから
  \jisho{}

  \z{答}{こた}えを\z{探}{さが}すたび\z{失}{うしな}うんだ
  \jisho{}

  \z{大事}{だいじ}なもの こぼれ\z{落}{お}ちていく
  \jisho{}

\item
  \z{幾千}{いくせん}の\z{時}{とき}を\z{超}{こ}えいつかまた\z{出会}{であ}う
  \jisho{}

  \z{繋}{つな}いだ\z{手}{て}の\z{感触}{かんしょく}を\z{思}{おも}い\z{出}{だ}して
  \jisho{}

  あの\z{夜}{よる}に\z{僕}{ぼく}らは\z{明日}{あした}を\z{願}{ねが}った
  \jisho{}

  \z{叶}{かな}わぬ\z{願}{ねが}いだとわかっていたとしても
  \jisho{}

\item
  \z{時}{とき}に\z{雲}{くも} \z{時}{とき}に\z{風}{かぜ} \z{形}{かたち}を\z{変}{か}えながら
  \jisho{}

  あなたの\z{元}{もと}に ほら \z{僕}{ぼく}は\z{向}{む}かうよ
  \jisho{}

\item
  そして\z{僕}{ぼく}の\z{声}{こえ}があなたに\z{届}{とど}くなら
  \jisho{}

  なんてあなたは\z{答}{こた}えるのだろう
  \jisho{}

  ありがとう ごめんね
  \jisho{}

  ひどいやつだ バカだな
  \jisho{}

  \z{愛}{あい}してる \z{泣}{な}いて\z{笑}{わら}うのは
  \jisho{}

  \z{多分}{たぶん}\z{僕}{ぼく}かも
  \jisho{}

  \z{聞}{き}こえる?
  \jisho{}

\item
  \z{悲}{かな}しみ \z{喜}{よろこ}び \z{心臓}{しんぞう}の\z{鼓動}{こどう}
  \jisho{}

  \z{伝}{つた}って\z{動}{うご}かすんだ \z{僕}{ぼく}という\z{命}{いのち}
  \jisho{}

  \z{想}{おも}いや\z{感情}{かんじょう} \z{掛}{か}け\z{値}{ね}なしの\z{愛}{あい}を
  \jisho{}

  あなたはくれたんだ
  \jisho{}

  \z{奇跡}{きせき}のような\z{日々}{ひび}を
  \jisho{}

  いつでもここにいるよ
  \jisho{}


}
\end{document}

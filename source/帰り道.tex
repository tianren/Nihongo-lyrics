% !TeX encoding = UTF-8
% !TeX program = LuaLaTeX

%\documentclass[12pt]{article}
\documentclass[14pt]{ltjsarticle}
% !TeX encoding = UTF-8
% !TeX program = LuaLaTeX

% !TeX encoding = UTF-8
% !TeX program = LuaLaTeX

%\documentclass[12pt]{article}

%\usepackage{luatexja-preset}
%\usepackage{fontspec}
\usepackage[no-math]{luatexja-fontspec}
\usepackage{luatexja-ruby}
\usepackage{color}

\topmargin=-0.95in      %
\evensidemargin=0in     %
\oddsidemargin=0in      %
\textwidth=6.5in        %
\textheight=9.75in       %
\headsep=0.25in

\setmainfont{Noto Sans CJK JP}
\setmainjfont{Noto Sans CJK JP}

\makeatletter
\def\dynscriptsize{\check@mathfonts\fontsize{\sf@size}{\z@}\selectfont}
\makeatother
\def\textunderset#1#2{\leavevmode
  \vtop{\offinterlineskip\halign{%
    \hfil##\hfil\cr\strut#2\cr\noalign{\kern-.3ex}
    \hidewidth\dynscriptsize\strut#1\hidewidth\cr}}}
\newcommand{\righttext}[1]{\ifx&#1&\else
  \hspace{1em} \unskip
  \nobreak \hfil \penalty1000 \hfilneg \indent
  \strut\hfill \mbox{\def\z{\zhuyin}\small #1}
  \fi}

\newcommand{\zhuyin}[2]{\ruby{#1}{#2}}
\newcommand{\kanji}[2]{\textunderset{#1}{\textcolor{blue}{#2}}}
\def\jisho{\righttext}
\def\trans{\righttext}

\newcommand{\lyrics}[1]{\begingroup
\def\z{\zhuyin}
\list{}{\leftmargin0em}
\setlength\itemsep{1em}
  #1
\endlist
\endgroup}

\makeatletter
\newcommand{\itemmark}[2]{%
\item[#1.]
\def\@currentlabel{#1}\label{par:#1}
  \expandafter\def\csname firstline@#1\endcsname{#2}%
  #2
}
\newcommand{\itemrepeat}[1]{
  \item[\ref{par:#1}.]
  \csname firstline@#1\endcsname
}
\makeatother


\renewcommand{\jisho}[1]{}

\renewcommand{\lyrics}[1]{\begingroup
\def\z{\zhuyin}
\list{}{\leftmargin0em}
\setlength\itemsep{1em}
  #1
\endlist
\endgroup
\clearpage
\begingroup
\def\z{\kanji}
\list{}{\leftmargin0em}
\setlength\itemsep{1em}
  #1
\endlist
\endgroup}

\begin{document}

\lyrics{
\item
  \textbf{帰り道 \hfill %
    「化物語」のOP}

\item
  ここをみぎ つぎひだり
  \jisho{}

  もぅ にっちもさっちも
  \jisho{}

  いかなくって \z{困}{こま}る
  \jisho{}

\item
  みぎひだり ひぎみだり
  \jisho{}

  バスの\z{窓}{まど}から
  \jisho{}

  \z{手}{て}や\z{足}{あし}を\z{出}{だ}すな
  \jisho{}

\item
  そこひだり すぐみぎへ
  \jisho{}

  もぅ どっちがどっちでも
  \jisho{}

  \z{結局}{けっきょく} \z{同}{おな}じ
  \jisho{}

\item
  \z{晴}{は}れのち\z{曇}{くも}り
  \jisho{}

  ところにより
  \jisho{}

  ときどき\z{雨}{あめ}みたいなことだよね
  \jisho{}

\item
  \z{寄}{よ}り\z{道}{みち}ばかりしてたら
  \jisho{}

  いつの\z{間}{ま}にか
  \jisho{}

  \z{日}{ひ}が\z{暮}{く}れてる
  \jisho{}

  \z{年}{とし}も\z{暮}{く}れてる
  \jisho{}

  \z{途方}{とほう}に\z{暮}{く}れちゃってる
  \jisho{}

\item
  お\z{腹}{なか}が\z{鳴}{な}くから\z{帰}{かえ}ろう
  \jisho{}

  まっすぐお\z{家}{うち}へ\z{帰}{かえ}ろう
  \jisho{}

  \z{心}{こころ}が\z{迷}{まよ}う\z{時}{とき}は
  \jisho{}

  その\z{笑顔}{えがお}が\z{目印}{めじるし}
  \jisho{}

\item
  \z{手}{て}と\z{手}{て}をつないで\z{帰}{かえ}ろう
  \jisho{}

  いっしょにお\z{家}{うち}へ\z{帰}{かえ}ろう
  \jisho{}

  いつだって そこにいて
  \jisho{}

  \z{見}{み}つけてくれる あなたと
  \jisho{}

\item
  \z{遠回}{とおまわ}りでも
  \jisho{}

  \z{遠回}{とおまわ}りじゃない
  \jisho{}

\item
  これひだり またひだり
  \jisho{}

  もぅ そっちはそっちで
  \jisho{}

  \z{目}{め}がまわってしまう
  \jisho{}

\item
  みぎをみて ひだりみて
  \jisho{}

  もいちど みぎひだり
  \jisho{}

  \z{無限}{むげん}ループ
  \jisho{}

\item
  まだひだり やっとみぎ
  \jisho{}

  もぅ あっちもこっちも
  \jisho{}

  うるさいな ちょっと
  \jisho{}

\item
  \z{曲}{ま}がるかと みせかけて
  \jisho{}

  \z{実}{じつ}はみちなりに
  \jisho{}

  まっすぐかもよ
  \jisho{}

\item
  \z{寄}{よ}り\z{道}{みち}ばかりしてたら
  \jisho{}

  いつの\z{間}{ま}にか
  \jisho{}

  \z{日}{ひ}が\z{暮}{く}れてる
  \jisho{}

  \z{年}{とし}も\z{暮}{く}れてる
  \jisho{}

  \z{途方}{とほう}に\z{暮}{く}れちゃってる
  \jisho{}

\item
  お\z{腹}{なか}が\z{鳴}{な}くから\z{帰}{かえ}ろう
  \jisho{}

  まっすぐお\z{家}{うち}へ\z{帰}{かえ}ろう
  \jisho{}

  \z{心}{こころ}が\z{迷}{まよ}う\z{時}{とき}は
  \jisho{}

  その\z{笑顔}{えがお}が\z{目印}{めじるし}
  \jisho{}

\item
  \z{手}{て}と\z{手}{て}をつないで\z{帰}{かえ}ろう
  \jisho{}

  いっしょにお\z{家}{うち}へ\z{帰}{かえ}ろう
  \jisho{}

  いつだって そこにいて
  \jisho{}

  \z{見}{み}つけてくれる あなたと
  \jisho{}

\item
  \z{遠回}{とおまわ}りでも
  \jisho{}

  \z{遠回}{とおまわ}りじゃない
  \jisho{}

\item
  \z{探}{さが}し\z{物}{もの}なら
  \jisho{}

  とっくに\z{見}{み}つけたけど
  \jisho{}

  どぅか\z{今}{いま}は
  \jisho{}

  このままでいて
  \jisho{}

  \z{眠}{ねむ}くなる\z{時間}{じかん}まで
  \jisho{}

  あと\z{少}{すこ}し
  \jisho{}

\item
  お\z{腹}{なか}が\z{鳴}{な}くから\z{帰}{かえ}ろう
  \jisho{}

  まっすぐお\z{家}{うち}へ\z{帰}{かえ}ろう
  \jisho{}

  \z{心}{こころ}が\z{迷}{まよ}う\z{時}{とき}は
  \jisho{}

  その\z{笑願}{えがお}が\z{目印}{めじるし}
  \jisho{}

\item
  \z{手}{て}と\z{手}{て}をつないで\z{帰}{かえ}ろう
  \jisho{}

  いっしょにお\z{家}{うち}へ\z{帰}{かえ}ろう
  \jisho{}

  いつだって そこにいて
  \jisho{}

  \z{見}{み}つけてくれる あなたと
  \jisho{}

  \z{遠回}{とおまわ}りでも
  \jisho{}

  \z{特別}{とくべつ}な\z{道}{みち}
  \jisho{}

\item
  \z{遠回}{とおまわ}りでも
  \jisho{}

  \z{遠回}{とおまわ}りじゃない
  \jisho{}

  
}
\end{document}


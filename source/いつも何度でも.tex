% !TeX encoding = UTF-8
% !TeX program = LuaLaTeX

%\documentclass[12pt]{article}
\documentclass[14pt]{ltjsarticle}
% !TeX encoding = UTF-8
% !TeX program = LuaLaTeX

% !TeX encoding = UTF-8
% !TeX program = LuaLaTeX

%\documentclass[12pt]{article}

%\usepackage{luatexja-preset}
%\usepackage{fontspec}
\usepackage[no-math]{luatexja-fontspec}
\usepackage{luatexja-ruby}
\usepackage{color}

\topmargin=-0.95in      %
\evensidemargin=0in     %
\oddsidemargin=0in      %
\textwidth=6.5in        %
\textheight=9.75in       %
\headsep=0.25in

\setmainfont{Noto Sans CJK JP}
\setmainjfont{Noto Sans CJK JP}

\makeatletter
\def\dynscriptsize{\check@mathfonts\fontsize{\sf@size}{\z@}\selectfont}
\makeatother
\def\textunderset#1#2{\leavevmode
  \vtop{\offinterlineskip\halign{%
    \hfil##\hfil\cr\strut#2\cr\noalign{\kern-.3ex}
    \hidewidth\dynscriptsize\strut#1\hidewidth\cr}}}
\newcommand{\righttext}[1]{\ifx&#1&\else
  \hspace{1em} \unskip
  \nobreak \hfil \penalty1000 \hfilneg \indent
  \strut\hfill \mbox{\def\z{\zhuyin}\small #1}
  \fi}

\newcommand{\zhuyin}[2]{\ruby{#1}{#2}}
\newcommand{\kanji}[2]{\textunderset{#1}{\textcolor{blue}{#2}}}
\def\jisho{\righttext}
\def\trans{\righttext}

\newcommand{\lyrics}[1]{\begingroup
\def\z{\zhuyin}
\list{}{\leftmargin0em}
\setlength\itemsep{1em}
  #1
\endlist
\endgroup}

\makeatletter
\newcommand{\itemmark}[2]{%
\item[#1.]
\def\@currentlabel{#1}\label{par:#1}
  \expandafter\def\csname firstline@#1\endcsname{#2}%
  #2
}
\newcommand{\itemrepeat}[1]{
  \item[\ref{par:#1}.]
  \csname firstline@#1\endcsname
}
\makeatother


\renewcommand{\jisho}[1]{}

\renewcommand{\lyrics}[1]{\begingroup
\def\z{\zhuyin}
\list{}{\leftmargin0em}
\setlength\itemsep{1em}
  #1
\endlist
\endgroup
\clearpage
\begingroup
\def\z{\kanji}
\list{}{\leftmargin0em}
\setlength\itemsep{1em}
  #1
\endlist
\endgroup}

\begin{document}

\lyrics{
\item
  \textbf{いつも\z{何度}{なんど}でも \hfill %
    「\z{千}{せん}と\z{千尋}{ちひろ}の\z{神隠}{かみかく}し」の ED}

\item
  \z{呼}{よ}んでいる \z{胸}{むね}のどこか\z{奥}{おく}で
  \jisho{}

  いつも\z{心}{こころ}\z{踊}{おど}る \z{夢}{ゆめ}を\z{見}{み}たい
  \jisho{}

\item
  かなしみは \z{数}{かぞ}えきれないけれど
  \jisho{}

  その\z{向}{む}こうできっと あなたに\z{会}{あ}える
  \jisho{}

\item
  \z{繰}{く}り\z{返}{かえ}すあやまちの そのたび ひとは
  \jisho{}

  ただ\z{青}{あお}い\z{空}{そら}の \z{青}{あお}さを\z{知}{し}る
  \jisho{}

  \z{果}{は}てしなく \z{道}{みち}は\z{続}{つづ}いて\z{見}{み}えるけれど
  \jisho{}

  この\z{両手}{りょうて}は \z{光}{ひかり}を\z{抱}{だ}ける
  \jisho{}

\item
  さよならのときの \z{静}{しず}かな\z{胸}{むね}
  \jisho{}

  ゼロになるからだが \z{耳}{みみ}をすませる
  \jisho{}

\item
  \z{生}{い}きている\z{不思議}{ふしぎ} \z{死}{し}んでいく\z{不思議}{ふしぎ}
  \jisho{}

  \z{花}{はな}も\z{風}{かぜ}も\z{街}{まち}も みんなおなじ
  \jisho{}

\item
  \z{呼}{よ}んでいる \z{胸}{むね}のどこか\z{奥}{おく}で
  \jisho{}

  いつも\z{何度}{なんど}でも \z{夢}{ゆめ}を\z{描}{えが}こう
  \jisho{}

\item
  かなしみの\z{数}{かず}を \z{言}{い}い\z{尽}{つ}くすより
  \jisho{}

  \z{同}{おな}じくちびるで そっと\z{歌}{うた}おう
  \jisho{}

\item
  \z{閉}{と}じていく\z{思}{おも}い\z{出}{で}の そのなかにいつも
  \jisho{}

  \z{忘}{わす}れたくない ささやきを\z{聞}{き}く
  \jisho{}

  こなごなに\z{砕}{くだ}かれた \z{鏡}{かがみ}の\z{上}{うえ}にも
  \jisho{}

  \z{新}{あたら}しい\z{景色}{けしき}が \z{映}{うつ}される
  \jisho{}

\item
  はじまりの\z{朝}{あさ}の \z{静}{しず}かな\z{窓}{まど}
  \jisho{}

  ゼロになるからだ \z{充}{み}たされてゆけ
  \jisho{}

\item
  \z{海}{うみ}の\z{彼方}{かなた}には もう\z{探}{さが}さない
  \jisho{}

  \z{輝}{かがや}くものは いつもここに
  \jisho{}

  わたしのなかに \z{見}{み}つけられたから
  \jisho{}


}
\end{document}

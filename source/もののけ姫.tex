% !TeX encoding = UTF-8
% !TeX program = LuaLaTeX

%\documentclass[12pt]{article}
\documentclass[14pt]{ltjsarticle}
../macros/dual-lyrics-luamacros.tex
\begin{document}

\lyrics{
\item
  \textbf{もののけ\z{姫}{ひめ} \hfill %
    「もののけ\z{姫}{ひめ}」の\zhuyin{主}{しゅ}\zhuyin{題}{だい}\zhuyin{歌}{か}}

\item
  はりつめた\z{弓}{ゆみ}の \z{震}{ふる}える\z{弦}{つる}よ
  \jisho{\z{張}{は}り\z{詰}{つ}める vi.\ to stretch, \z{震}{ふる}える vi.\ to shiver}

  \z{月}{つき}の\z{光}{ひかり}にざわめく お\z{前}{まえ}の\z{心}{こころ}
  \jisho{\z{騒}{ざわ}めく to be noisy}

\item
  とぎすまされた \z{刃}{やいば}の\z{美}{うつく}しい
  \jisho{\z{研}{と}ぎ\z{澄}{す}ます to sharpen}

  そのきっさきによく\z{似}{に}た そなたの\z{横顔}{よこがお}
  \jisho{\z{切}{き}っ\z{先}{さき} point (of a sword), \z{似}{に}る to resemble}

\item
  \z{悲}{かな}しみと\z{怒}{いか}りにひそむ \z{誠}{まこと}の\z{心}{こころ}を\z{知}{し}るは
  \jisho{\z{潜}{ひそ}む to lurk; to be hidden}

  \z{森}{もり}の\z{精}{せい} もののけ\z{達}{たち}だけ もののけ\z{達}{たち}だけ
  \jisho{\z{物}{もの}の\z{怪}{け} ghost}

  
}
\end{document}


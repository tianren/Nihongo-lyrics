% !TeX encoding = UTF-8
% !TeX program = LuaLaTeX

%\documentclass[12pt]{article}
\documentclass[14pt]{ltjsarticle}
../macros/dual-lyrics-luamacros.tex
\begin{document}

\lyrics{
\item
  \textbf{言わないけどね。 \hfill %
    「からかい上手の高木さん」のOP}

\item
  \z{勘違}{かんちが}いされちゃったっていいよ
  \jisho{}

  \z{君}{きみ}とならなんて \z{思}{おも}ってったって\z{言}{い}わないけどね
  \jisho{}

  \z{近}{ちか}づく\z{空}{そら}の\z{香}{かお}りを \z{隣}{となり}で\z{感}{かん}じていたいの
  \jisho{}

\item
  \z{校庭}{こうてい}で\z{君}{きみ}のことを\z{一番}{いちばん}に\z{見}{み}つけて\z{今日}{きょう}は
  \jisho{}

  なんて\z{話}{はな}しかけようか ちょっと\z{考}{かんが}えるの\z{楽}{たの}しくて
  \jisho{}

\item
  \z{風}{かぜ}に\z{揺}{ゆ}らされるカーテン \z{不意打}{ふいう}ちに\z{当}{あ}たる\z{日差}{ひざ}し
  \jisho{}

  \z{眩}{まぶ}しそうな\z{顔}{かお}を\z{笑}{わら}ったら \z{照}{て}れて\z{伏}{ふ}せちゃうのね
  \jisho{}

\item
  \z{憂}{ゆう}うつなテストも \z{吹}{ふ}き\z{飛}{と}ばせるような
  \jisho{}

  ねえ \z{二人}{ふたり}で\z{秘密}{ひみつ}の\z{約束}{やくそく}をしたいなぁ って\z{提案}{ていあん}です
  \jisho{}

\item
  \z{勘違}{かんちが}いされちゃったっていいよ
  \jisho{}

  \z{君}{きみ}とならなんて \z{思}{おも}ってったって\z{言}{い}わないけどね
  \jisho{}

  \z{近付}{ちかづ}く\z{空}{そら}の\z{香}{かお}りを \z{隣}{となり}で\z{感}{かん}じていたいの
  \jisho{}

\item
  \z{席替}{せきが}えが\z{嫌}{いや}だなんて \z{思}{おも}うのは\z{私}{わたし}だけかな
  \jisho{}

  \z{君}{きみ}の\z{隣}{となり}じゃないなら きっと\z{少}{すこ}し\z{退屈}{たいくつ}な\z{日々}{ひび}ね
  \jisho{}

\item
  \z{可愛}{かわい}くない\z{落書}{らくが}きや \z{真剣}{しんけん}な\z{表情}{ひょうじょう}にも
  \jisho{}

  \z{気付}{きづ}けないなんて\z{嫌}{いや}なのよ それだけじゃないけど
  \jisho{}

\item
  \z{外}{そと}を\z{眺}{なが}めるフリ \z{横顔}{よこがお}を\z{見}{み}ていた
  \jisho{}

  ねえ \z{君}{きみ}の\z{心}{こころ}の\z{中}{なか}\z{覗}{のぞ}いてみたいなぁ って\z{思}{おも}ってます
  \jisho{}

\item
  \z{勘違}{かんちが}いさせちゃったっていいの
  \jisho{}

  \z{特別}{とくべつ}だなんて \z{思}{おも}ってったって\z{言}{い}わないけどね
  \jisho{}

  \z{不思議}{ふしぎ}なままの\z{関係}{かんけい} \z{変}{か}われる\z{時}{とき}は \z{来}{く}るのかなぁ
  \jisho{}

\item
  \z{学校}{がっこう}じゃ\z{話}{はな}せない\z{事}{こと}もいっぱいあるの
  \jisho{}

  それが\z{何}{なん}なのか \z{知}{し}りたいなら\z{私}{わたし}と
  \jisho{}

\item
  いつか\z{制服}{せいふく}じゃない
  \jisho{}

  \z{君}{きみ}の\z{事}{こと}もっと \z{見}{み}たいなんて\z{言}{い}わないけどね
  \jisho{}

  \z{会}{あ}いたいの\z{代}{か}わりの\z{言葉}{ことば} \z{探}{さが}しているの
  \jisho{}

\item
  \z{勘違}{かんちが}いされちゃってもいいよ
  \jisho{}

  \z{君}{きみ}とならなんて \z{思}{おも}ってったって\z{言}{い}わないけどね
  \jisho{}

  \z{机}{つくえ}の\z{距離}{きょり}よりもっと \z{近}{ちか}くに\z{感}{かん}じていたいの
  \jisho{}

  \z{君}{きみ}をね
  \jisho{}

}
\end{document}


% !TeX encoding = UTF-8
% !TeX program = LuaLaTeX

%\documentclass[12pt]{article}
\documentclass[14pt]{ltjsarticle}
% !TeX encoding = UTF-8
% !TeX program = LuaLaTeX

% !TeX encoding = UTF-8
% !TeX program = LuaLaTeX

%\documentclass[12pt]{article}

%\usepackage{luatexja-preset}
%\usepackage{fontspec}
\usepackage[no-math]{luatexja-fontspec}
\usepackage{luatexja-ruby}
\usepackage{color}

\topmargin=-0.95in      %
\evensidemargin=0in     %
\oddsidemargin=0in      %
\textwidth=6.5in        %
\textheight=9.75in       %
\headsep=0.25in

\setmainfont{Noto Sans CJK JP}
\setmainjfont{Noto Sans CJK JP}

\makeatletter
\def\dynscriptsize{\check@mathfonts\fontsize{\sf@size}{\z@}\selectfont}
\makeatother
\def\textunderset#1#2{\leavevmode
  \vtop{\offinterlineskip\halign{%
    \hfil##\hfil\cr\strut#2\cr\noalign{\kern-.3ex}
    \hidewidth\dynscriptsize\strut#1\hidewidth\cr}}}
\newcommand{\righttext}[1]{\ifx&#1&\else
  \hspace{1em} \unskip
  \nobreak \hfil \penalty1000 \hfilneg \indent
  \strut\hfill \mbox{\def\z{\zhuyin}\small #1}
  \fi}

\newcommand{\zhuyin}[2]{\ruby{#1}{#2}}
\newcommand{\kanji}[2]{\textunderset{#1}{\textcolor{blue}{#2}}}
\def\jisho{\righttext}
\def\trans{\righttext}

\newcommand{\lyrics}[1]{\begingroup
\def\z{\zhuyin}
\list{}{\leftmargin0em}
\setlength\itemsep{1em}
  #1
\endlist
\endgroup}

\makeatletter
\newcommand{\itemmark}[2]{%
\item[#1.]
\def\@currentlabel{#1}\label{par:#1}
  \expandafter\def\csname firstline@#1\endcsname{#2}%
  #2
}
\newcommand{\itemrepeat}[1]{
  \item[\ref{par:#1}.]
  \csname firstline@#1\endcsname
}
\makeatother


\renewcommand{\jisho}[1]{}

\renewcommand{\lyrics}[1]{\begingroup
\def\z{\zhuyin}
\list{}{\leftmargin0em}
\setlength\itemsep{1em}
  #1
\endlist
\endgroup
\clearpage
\begingroup
\def\z{\kanji}
\list{}{\leftmargin0em}
\setlength\itemsep{1em}
  #1
\endlist
\endgroup}

\begin{document}

\lyrics{
\item
  \textbf{White Wishes \hfill %
    「となりの怪物くん」のED}

\item
  \z{星}{ほし}の\z{街路樹}{がいろじゅ} \z{君}{きみ}と\z{歩}{ある}く\z{街}{まち}
  \jisho{}

  \z{笑}{わら}った\z{途端}{とたん} \z{白}{しろ}い\z{息}{いき}が \z{凍}{こご}えそう
  \jisho{}

\item
  Ah どうして \z{今日}{きょう}も\z{冗談}{じょうだん}ばかりで
  \jisho{}

  \z{相変}{あいかわ}わらず \z{微妙}{びみょう}な\z{距離}{きょり} もどかしくて\z{切}{せつ}ないけど
  \jisho{}

\item
  めぐるめぐる \z{冬}{ふゆ}の\z{夜}{よる}も \z{今年}{ことし}はあったかいね
  \jisho{}

  \z{苦手}{にがて}だった \z{寒}{さむ}さだって なんだかうれしくて
  \jisho{}

  「\z{遠回}{とおまわ}りしよう」\z{君}{きみ}から \z{言}{い}わないかな
  \jisho{}

\item
  これが\z{恋}{こい}か \z{恋}{こい}じゃないか どっちだって\z{構}{かま}わない
  \jisho{}

  \z{不器用}{ぶきよう}でも ケンカしても やっぱり\z{君}{きみ}がいい
  \jisho{}

  \z{降}{ふ}り\z{出}{だ}した\z{雪}{ゆき}に \z{願}{ねが}いをかけるよ
  \jisho{}

  あとちょっと このままで ふたりきり
  \jisho{}

\item
  ふざけたふりで \z{指}{ゆび}が\z{触}{ふ}れた\z{時}{とき}
  \jisho{}

  \z{世界}{せかい}がふっと\z{色}{いろ}を\z{変}{か}えた その\z{瞬間}{しゅんかん}
  \jisho{}

\item
  Ah ちょうどいいサイズ \z{君}{きみ}と\z{私}{わたし}の\z{手}{て}
  \jisho{}

  こんなふうに \z{繋}{つな}ぐことが とっくに\z{決}{き}まってたみたい
  \jisho{}

\item
  めくるめくる \z{冬}{ふゆ}のページ \z{泣}{な}いたり\z{笑}{わら}ったり
  \jisho{}

  \z{思}{おも}いがけず ふたりだけの \z{秘密}{ひみつ}が\z{増}{ふ}えて\z{行}{い}く
  \jisho{}

  いつもより\z{胸}{むね}の\z{鼓動}{こどう}が うるさいけど
  \jisho{}

\item
  これが\z{恋}{こい}か \z{恋}{こい}じゃないか どっちだって\z{構}{かま}わない
  \jisho{}

  \z{一途}{いちず}すぎて はみ\z{出}{だ}しても やっぱり\z{君}{きみ}がいい
  \jisho{}

  \z{降}{ふ}りしきる\z{雪}{ゆき}に \z{願}{ねが}いをかけるよ
  \jisho{}

  あとちょっと このままで ふたりきり
  \jisho{}

\item
  めぐるめぐる \z{君}{きみ}と\z{私}{わたし} \z{今年}{ことし}も\z{来年}{らいねん}も
  \jisho{}

  \z{春}{はる}も\z{夏}{なつ}も \z{秋}{あき}も\z{冬}{ふゆ}も \z{一緒}{いっしょ}にいれるかな
  \jisho{}

  \z{素直}{すなお}に\z{言葉}{ことば}にできない \z{私}{わたし}だけど
  \jisho{}

\item
  これが\z{恋}{こい}か \z{恋}{こい}じゃないか \z{本当}{ほんとう}は\z{分}{わ}かってる
  \jisho{}

  \z{意地}{いじ}を\z{張}{は}って \z{強}{つよ}がっても やっぱり\z{君}{きみ}がいい
  \jisho{}

  \z{降}{ふ}り\z{積}{つ}もる\z{雪}{ゆき}に \z{願}{ねが}いをかけるよ
  \jisho{}

  ずっとずっと このままで そばにいて
  \jisho{}


}
\end{document}

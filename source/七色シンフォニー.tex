% !TeX encoding = UTF-8
% !TeX program = LuaLaTeX

%\documentclass[12pt]{article}
\documentclass[14pt]{ltjsarticle}
../macros/dual-lyrics-luamacros.tex
\begin{document}

\lyrics{
\item
  \textbf{七色シンフォニー \hfill %
    「四月は君の嘘」のOP2}

\item
  \z{今}{いま}\z{鮮}{あざ}やかなシンフォニー
  \jisho{}

  \z{七色}{なないろ}シンフォニー
  \jisho{}

\item
  \z{忘}{わす}れようとすることで \z{傷}{きず}が\z{癒}{い}えないのは
  \jisho{}

  \z{忘}{わす}れようとすることで \z{思}{おも}い\z{出}{だ}されるから
  \jisho{}

  \z{僕}{ぼく}は \z{巡}{めぐ}り\z{巡}{めぐ}り\z{巡}{めぐ}り\z{巡}{めぐ}り\z{巡}{めぐ}ってく
  \jisho{}

  \z{止}{と}まった\z{時計}{とけい}の\z{前}{まえ}で \z{立}{た}ちつくすのはやめよう
  \jisho{}

\item

いつまでも \z{君}{きみ}といたいと
  \jisho{}

  \z{強}{つよ}く\z{強}{つよ}く\z{思}{おも}うほど
  \jisho{}


いてもたっても いられなくなるよ
  \jisho{}

  \z{僕}{ぼく}は\z{雨}{あめ} \z{君}{きみ}は\z{太陽}{たいよう} \z{手}{て}を\z{繋}{つな}ごう
  \jisho{}

  \z{僕}{ぼく}らはここにいる
  \jisho{}

\item
  \z{今}{いま}\z{鮮}{あざ}やかなシンフォニー
  \jisho{}

  \z{七色}{なないろ}シンフォニー
  \jisho{}


ひとりじゃ\z{出}{だ}せない\z{音}{おと}が
  \jisho{}


あることに\z{気}{き}が\z{付}{つ}いたよ
  \jisho{}

  \z{泣}{な}いて\z{笑}{わら}って ドレミファソ
  \jisho{}

  \z{想}{おも}い\z{響}{ひび}き\z{合}{あ}うシンフォニー
  \jisho{}

\item
  \z{白}{しろ}いため\z{息}{いき}は いつの\z{間}{ま}にか\z{空}{そら}に\z{消}{き}えて
  \jisho{}

  \z{見上}{みあ}げれば \z{桜}{さくら}はピンクのつぼみをつける
  \jisho{}

  \z{僕}{ぼく}は \z{巡}{めぐ}り\z{巡}{めぐ}り\z{巡}{めぐ}り\z{巡}{めぐ}り\z{巡}{めぐ}ってく
  \jisho{}

  \z{喜}{よろこ}びも\z{切}{せつ}なさも\z{背負}{せお}って \z{春}{はる}を\z{待}{ま}っている
  \jisho{}

\item
  \z{不思議}{ふしぎ}だよ \z{君}{きみ}の\z{笑顔}{えがお}は
  \jisho{}


モノクロームの\z{街}{まち}を
  \jisho{}

  \z{色鮮}{いろあざ}やかに \z{染}{そ}めてゆくんだ
  \jisho{}


ねぇ \z{今}{いま}この\z{一瞬}{いっしゅん}を\z{抱}{だ}きしめよう
  \jisho{}

  \z{僕}{ぼく}らはここにいる
  \jisho{}

\item
  \z{空}{そら}に\z{花}{はな}びらひらり
  \jisho{}

  \z{春色}{はるいろ}シンフォニー
  \jisho{}

  \z{今}{いま}しか\z{出}{だ}せない\z{音}{おと}が
  \jisho{}


あることに\z{気}{き}が\z{付}{つ}いたよ
  \jisho{}

  \z{君}{きみ}がいるから \z{笑}{わら}えるよ
  \jisho{}

  \z{時}{とき}を\z{分}{わ}かち\z{合}{あ}うシンフォニー
  \jisho{}

\item
  \z{君}{きみ}はいつも \z{魔法使}{まほうつか}い
  \jisho{}

  \z{普通}{ふつう}の\z{日々}{ひび}のメロディー
  \jisho{}


そのすべてを \z{名曲}{めいきょく}にするんだ
  \jisho{}


そう まるでチャイコフスキー
  \jisho{}

  \z{勇気}{ゆうき}に\z{満}{み}ちた\z{音}{おと}をくれるんだ
  \jisho{}

\item
  \z{今}{いま}\z{鮮}{あざ}やかなシンフォニー
  \jisho{}

  \z{七色}{なないろ}シンフォニー
  \jisho{}


ひとりじゃ\z{出}{だ}せない\z{音}{おと}が
  \jisho{}


あることに\z{気}{き}が\z{付}{つ}いたよ
  \jisho{}

  \z{駆}{か}け\z{上}{あ}がるように ドレミファソ
  \jisho{}

  \z{僕}{ぼく}が\z{奏}{かな}でてるメロディー
  \jisho{}

  \z{君}{きみ}が\z{奏}{かな}でてるメロディー
  \jisho{}

  \z{想}{おも}い\z{響}{ひび}き\z{合}{あ}うシンフォニー
  \jisho{}


}
\end{document}

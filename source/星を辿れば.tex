% !TeX encoding = UTF-8
% !TeX program = LuaLaTeX

%\documentclass[12pt]{article}
\documentclass[14pt]{extreport}
../macros/dual-lyrics-luamacros.tex
\begin{document}

\lyrics{
\item
  \textbf{星を辿れば \hfill 「Little Witch Academia」のED}

\item
  \z{夢中}{むちゅう}になって\z{追}{お}いかけてた
  \jisho{\z{夢中}{むちゅう}になって=\z{我}{われ}を\z{忘}{わす}れて; %
    \z{追}{お}い\z{駆}{か}ける [ru] to run after}

  \z{流}{なが}れ\z{星}{ぼし} \z{辿}{たど}った\z{先}{さき}に
  \jisho{\z{辿}{たど}る to follow; \z{先}{さき}に before}

  \z{広}{ひろ}がる\z{世界}{せかい}を\z{知}{し}りたくて
  \jisho{\z{広}{ひろ}がる to extend}

\item
  \z{叶}{かな}えたいと \z{願}{ねが}うだけじゃない
  \jisho{\z{叶}{かな}える [ru] to fulfill; %
    たい \z{希}{き}\z{望}{ぼう}の\z{助}{じょ}\z{動}{どう}\z{詞}{し}; %
    だけじゃない not only}

  \z{扉}{とびら}を\z{開}{あ}ければ ほら \z{近}{ちか}づいていけるよ
  \jisho{\z{開}{あ}ける [ru] to open; \z{近}{ちか}づく to approach}

\item
  \z{待}{ま}ちきれない \z{物語}{ものがたり}の\z{続}{つづ}きを\z{探}{さが}そう
  \jisho{\z{切}{き}れない being too much; \z{続}{つづ}き continuation; %
    \z{探}{さが}す to search}

  \z{未来}{みらい} \z{信}{しん}じられるのは \z{一人}{ひとり}じゃないから
  \jisho{}

  \z{景色}{けしき}は\z{変}{か}わっていくけど \z{輝}{かがや}く\z{空}{そら}と
  \jisho{\z{変}{か}わる to change}

  \z{巡}{めぐ}り\z{合}{あ}った \z{宝物}{たからもの} いつまでも\z{放}{はな}さないよ
  \jisho{\z{巡}{めぐ}り\z{合}{あ}う to meet by chance; %
    \z{何時}{いつ}までも forever; \z{放}{はな}す to release}

\item
  そう \z{泣}{な}いて\z{笑}{わら}って\z{食}{た}べて\z{寝}{ね}ても
  \jisho{\z{泣}{な}く to cry; \z{笑}{わら}う to laugh}

  クヨクヨしちゃいそうな\z{時}{とき}でも
  \jisho{くよくよ worrying about; しちゃ=しては}

  \z{側}{そば}にいてくれて ありがとう
  \jisho{\z{暮}{く}れる [ru] to get dark, to end}

  また\z{明日}{あした}ね
  \jisho{}

\item
  \z{日記}{にっき}に\z{書}{か}ききれない\z{日々}{ひび}を
  \jisho{}

  \z{夜}{よる}の\z{空}{そら}に\z{描}{えが}いていたの
  \jisho{}

  \z{小}{ちい}さな\z{光}{ひかり} \z{繋}{つな}ぐ\z{星座}{せいざ}
  \jisho{}

\item
  \z{夢}{ゆめ}にはまだ \z{届}{とど}かないけど
  \jisho{}

  それでも\z{目指}{めざ}す\z{場所}{ばしょ}へ
  \jisho{}

  \z{真}{ま}っ\z{直}{す}ぐ\z{手}{て}を\z{伸}{の}ばすよ
  \jisho{}

\item
  \z{夢見}{ゆめみ}た\z{日}{ひ}の\z{光}{ひかり}\z{今}{いま}も \z{照}{て}らしてくれる
  \jisho{}

  \z{迷}{まよ}わないでいれるの どんな\z{暗闇}{くらやみ}でも
  \jisho{}

  \z{一人}{ひとり}\z{煌}{きら}めくだけじゃきっと\z{届}{とど}かないけど
  \jisho{}

  \z{巡}{めぐ}り\z{合}{あ}えた\z{特別}{とくべつ}な\z{言葉}{ことば}はいつも\z{力}{ちから}になるから
  \jisho{}

\item
  \z{賑}{にぎ}やかな \z{今日}{きょう}を\z{夜空}{よぞら}へ
  \jisho{}

  \z{浮}{う}かべて \z{眠}{ねむ}る\z{前}{まえ}に \z{見上}{みあ}げるよ
  \jisho{}

\item
  \z{散}{ち}りばめられた\z{鮮}{あざ}やかな\z{星}{ほし}たちのように
  \jisho{}

  おんなじものは \z{何一}{なにひと}つ \z{有}{あ}りはしないの
  \jisho{}

  どんな \z{歪}{いびつ}な\z{希望}{きぼう}だって \z{欠}{か}かせないから
  \jisho{}

  \z{巡}{めぐ}り\z{会}{あ}えた\z{奇跡}{きせき}は\z{時}{とき}を\z{越}{こ}えて\z{重}{かさ}なるよ
  \jisho{}

\item
  そう \z{喜}{よろこ}んで \z{怒}{いか}って \z{悩}{なや}んで
  \jisho{}

  クルクル\z{目}{め}まぐるしい\z{日々}{ひび}でも
  \jisho{}

  \z{側}{そば}に\z{居}{い}てくれてありがとう
  \jisho{}

  おやすみなさい
  \jisho{}

  また\z{明日}{あした}ね
  \jisho{}


}

\end{document}

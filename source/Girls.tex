% !TeX encoding = UTF-8
% !TeX program = LuaLaTeX

%\documentclass[12pt]{article}
\documentclass[14pt]{ltjsarticle}
% !TeX encoding = UTF-8
% !TeX program = LuaLaTeX

% !TeX encoding = UTF-8
% !TeX program = LuaLaTeX

%\documentclass[12pt]{article}

%\usepackage{luatexja-preset}
%\usepackage{fontspec}
\usepackage[no-math]{luatexja-fontspec}
\usepackage{luatexja-ruby}
\usepackage{color}

\topmargin=-0.95in      %
\evensidemargin=0in     %
\oddsidemargin=0in      %
\textwidth=6.5in        %
\textheight=9.75in       %
\headsep=0.25in

\setmainfont{Noto Sans CJK JP}
\setmainjfont{Noto Sans CJK JP}

\makeatletter
\def\dynscriptsize{\check@mathfonts\fontsize{\sf@size}{\z@}\selectfont}
\makeatother
\def\textunderset#1#2{\leavevmode
  \vtop{\offinterlineskip\halign{%
    \hfil##\hfil\cr\strut#2\cr\noalign{\kern-.3ex}
    \hidewidth\dynscriptsize\strut#1\hidewidth\cr}}}
\newcommand{\righttext}[1]{\ifx&#1&\else
  \hspace{1em} \unskip
  \nobreak \hfil \penalty1000 \hfilneg \indent
  \strut\hfill \mbox{\def\z{\zhuyin}\small #1}
  \fi}

\newcommand{\zhuyin}[2]{\ruby{#1}{#2}}
\newcommand{\kanji}[2]{\textunderset{#1}{\textcolor{blue}{#2}}}
\def\jisho{\righttext}
\def\trans{\righttext}

\newcommand{\lyrics}[1]{\begingroup
\def\z{\zhuyin}
\list{}{\leftmargin0em}
\setlength\itemsep{1em}
  #1
\endlist
\endgroup}

\makeatletter
\newcommand{\itemmark}[2]{%
\item[#1.]
\def\@currentlabel{#1}\label{par:#1}
  \expandafter\def\csname firstline@#1\endcsname{#2}%
  #2
}
\newcommand{\itemrepeat}[1]{
  \item[\ref{par:#1}.]
  \csname firstline@#1\endcsname
}
\makeatother


\renewcommand{\jisho}[1]{}

\renewcommand{\lyrics}[1]{\begingroup
\def\z{\zhuyin}
\list{}{\leftmargin0em}
\setlength\itemsep{1em}
  #1
\endlist
\endgroup
\clearpage
\begingroup
\def\z{\kanji}
\list{}{\leftmargin0em}
\setlength\itemsep{1em}
  #1
\endlist
\endgroup}

\begin{document}

\lyrics{
\item
  \textbf{Girls \hfill
    GARNiDELiA}

\item
  \z{人}{ひと}の\z{気持}{きも}ちを\z{知}{し}ったように\z{書}{か}いた
  \jisho{}

  
ビジネスライクな\z{雑誌}{ざっし}で\z{予習}{よしゅう}
  \jisho{}

  \z{言葉}{ことば}の\z{意味}{いみ}をわかってるのか
  \jisho{}

  
こっちが\z{心配}{しんぱい}になってくる
  \jisho{}

\item
  \z{大}{おお}きく\z{見}{み}せてるところは\z{可愛}{かわい}い
  \jisho{}

  
そのツケをどこにもってくるの?
  \jisho{}

\item
  \z{自慢話}{じまんばなし}を\z{聞}{き}きに\z{来}{き}たんじゃない
  \jisho{}

  
あなたは\z{私}{わたし}をどうしたいの?
  \jisho{}

  \z{時間}{じかん}だってねぇ タダじゃないんだし
  \jisho{}

  \z{響}{ひび}かない\z{言葉並}{ことばなら}べないで
  \jisho{}

  \z{惨}{みじ}めなところ\z{見}{み}せないで
  \jisho{}

  
このまま\z{帰}{かえ}してもいいの?
  \jisho{}

\item
  
「\z{恋}{こい}の\z{始}{はじ}まりは\z{理屈}{りくつ}じゃないよ」
  \jisho{}

  \z{全然始}{ぜんぜんはじ}まりそうになんかない
  \jisho{}

  
「\z{熱}{ねっ}しやすくて\z{冷}{さ}めやすいんだ、オレ」
  \jisho{}

  
そんなこと\z{誰}{だれ}も\z{聞}{き}いてないでしょ
  \jisho{}

\item
  
ムダに\z{共通点探}{きょうつうてんさぐ}ろうとしてる
  \jisho{}

  
がっかりさせないでもうこれ\z{以上}{いじょう}
  \jisho{}

\item
  \z{追}{お}いかけたくなるくらいにさせてよ
  \jisho{}

  
あなたは\z{私}{わたし}をどうしたいの?
  \jisho{}

  \z{行}{い}くあてもない\z{心}{こころ}が\z{寂}{さび}しい
  \jisho{}

  \z{響}{ひび}かない\z{心}{こころ}を\z{動}{うご}かして
  \jisho{}

  \z{私}{わたし}だって\z{笑}{わら}ってたいの
  \jisho{}

  \z{誰}{だれ}かこの\z{扉}{とびら}を\z{開}{あ}けて
  \jisho{}

\item
  \z{寂}{さび}しさを\z{隠}{かく}してる\z{訳}{わけ}じゃなくて
  \jisho{}

  \z{静}{しず}かにさせてこんな\z{私}{わたし}を
  \jisho{}

\item
  \z{自慢話}{じまんばなし}を\z{聞}{き}きに\z{来}{き}たんじゃない
  \jisho{}

  
あなたは\z{私}{わたし}をどうしたいの?
  \jisho{}

  \z{時間}{じかん}だってねぇ タダじゃないんだし
  \jisho{}

  \z{響}{ひび}かない\z{言葉並}{ことばなら}べないで
  \jisho{}

\item
  \z{追}{お}いかけたくなるくらいにさせてよ
  \jisho{}

  
あなたは\z{私}{わたし}をどうしたいの?
  \jisho{}

  \z{行}{い}くあてもない\z{心}{こころ}が\z{寂}{さび}しい
  \jisho{}

  \z{響}{ひび}かない\z{心}{こころ}を\z{動}{うご}かして
  \jisho{}

  \z{私}{わたし}だって\z{笑}{わら}ってたいの
  \jisho{}

  \z{誰}{だれ}かこの\z{扉}{とびら}を\z{開}{あ}けて
  \jisho{}

\item
  \z{惨}{みじ}めなところ\z{見}{み}せないで
  \jisho{}

  
このまま\z{帰}{かえ}してもいいの?
  \jisho{}

  
}
\end{document}


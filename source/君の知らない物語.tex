% !TeX encoding = UTF-8
% !TeX program = LuaLaTeX

%\documentclass[12pt]{article}
\documentclass[14pt]{ltjsarticle}
% !TeX encoding = UTF-8
% !TeX program = LuaLaTeX

% !TeX encoding = UTF-8
% !TeX program = LuaLaTeX

%\documentclass[12pt]{article}

%\usepackage{luatexja-preset}
%\usepackage{fontspec}
\usepackage[no-math]{luatexja-fontspec}
\usepackage{luatexja-ruby}
\usepackage{color}

\topmargin=-0.95in      %
\evensidemargin=0in     %
\oddsidemargin=0in      %
\textwidth=6.5in        %
\textheight=9.75in       %
\headsep=0.25in

\setmainfont{Noto Sans CJK JP}
\setmainjfont{Noto Sans CJK JP}

\makeatletter
\def\dynscriptsize{\check@mathfonts\fontsize{\sf@size}{\z@}\selectfont}
\makeatother
\def\textunderset#1#2{\leavevmode
  \vtop{\offinterlineskip\halign{%
    \hfil##\hfil\cr\strut#2\cr\noalign{\kern-.3ex}
    \hidewidth\dynscriptsize\strut#1\hidewidth\cr}}}
\newcommand{\righttext}[1]{\ifx&#1&\else
  \hspace{1em} \unskip
  \nobreak \hfil \penalty1000 \hfilneg \indent
  \strut\hfill \mbox{\def\z{\zhuyin}\small #1}
  \fi}

\newcommand{\zhuyin}[2]{\ruby{#1}{#2}}
\newcommand{\kanji}[2]{\textunderset{#1}{\textcolor{blue}{#2}}}
\def\jisho{\righttext}
\def\trans{\righttext}

\newcommand{\lyrics}[1]{\begingroup
\def\z{\zhuyin}
\list{}{\leftmargin0em}
\setlength\itemsep{1em}
  #1
\endlist
\endgroup}

\makeatletter
\newcommand{\itemmark}[2]{%
\item[#1.]
\def\@currentlabel{#1}\label{par:#1}
  \expandafter\def\csname firstline@#1\endcsname{#2}%
  #2
}
\newcommand{\itemrepeat}[1]{
  \item[\ref{par:#1}.]
  \csname firstline@#1\endcsname
}
\makeatother


\renewcommand{\jisho}[1]{}

\renewcommand{\lyrics}[1]{\begingroup
\def\z{\zhuyin}
\list{}{\leftmargin0em}
\setlength\itemsep{1em}
  #1
\endlist
\endgroup
\clearpage
\begingroup
\def\z{\kanji}
\list{}{\leftmargin0em}
\setlength\itemsep{1em}
  #1
\endlist
\endgroup}

\begin{document}

\lyrics{
\item
  \textbf{君の知らない物語 \hfill %
    「化物語」のED}

\item
  いつもどおりのある\z{日}{ひ}の\z{事}{こと}
  \jisho{いつも\z{通}{どお}り as always}

  \z{君}{きみ}は\z{突然}{とつぜん}\z{立}{た}ち\z{上}{あ}がり\z{言}{い}った
  \jisho{}

  「\z{今夜}{こんや}\z{星}{ほし}を\z{見}{み}に\z{行}{ゆ}こう」
  \jisho{}

\item
  「たまには\z{良}{い}いこと\z{言}{ゆ}うんだね」
  \jisho{}

  なんてみんなして\z{言}{い}って\z{笑}{わら}った
  \jisho{}

  \z{明}{あ}かりもない\z{道}{みち}を
  \jisho{\z{明}{あ}かり light}

  バカみたいにはしゃいで\z{歩}{ある}いた
  \jisho{}

  \z{抱}{かか}え\z{込}{こ}んだ\z{孤独}{こどく}や\z{不安}{ふあん}に
  \jisho{}

  \z{押}{お}しつぶされないように
  \jisho{\z{押}{お}しつぶす to squash; to crush}

\item
  \z{真}{ま}っ\z{暗}{くら}な\z{世界}{せかい}から\z{見上}{みあ}げた
  \jisho{}

  \z{夜空}{よぞら}は\z{星}{ほし}が\z{降}{ふ}るようで
  \jisho{}

\item
  いつからだろう \z{君}{きみ}の\z{事}{こと}を
  \jisho{}

  \z{追}{お}いかける\z{私}{わたし}がいた
  \jisho{}

  どうかお\z{願}{ねが}い
  \jisho{}

  \z{驚}{おどろ}かないで\z{聞}{き}いてよ
  \jisho{}

  \z{私}{わたし}のこの\z{想}{おも}いを
  \jisho{}

\item
  「あれがデネブ、アルタイル、ベガ」
  \jisho{}

  \z{君}{きみ}は\z{指}{ゆび}さす\z{夏}{なつ}の\z{大三角}{だいさんかく}
  \jisho{}

  \z{覚}{おぼ}えて\z{空}{そら}を\z{見}{み}る
  \jisho{}

  やっと\z{見}{み}つけた\z{織姫様}{おりひめさま}
  \jisho{}

  だけどどこだろう\z{彦星様}{ひこぼしさま}
  \jisho{}

  これじゃひとりぼっち
  \jisho{}

\item
  \z{楽}{たの}しげなひとつ\z{隣}{となり}の\z{君}{きみ}
  \jisho{}

  \z{私}{わたし}は\z{何}{なに}も\z{言}{い}えなくて
  \jisho{}

\item
  \z{本当}{ほんとう}はずっと\z{君}{きみ}の\z{事}{こと}を
  \jisho{}

  どこかでわかっていた
  \jisho{}

  \z{見}{み}つかったって
  \jisho{}

  \z{届}{とど}きはしない
  \jisho{}

  だめだよ \z{泣}{な}かないで
  \jisho{}

  そう\z{言}{い}い\z{聞}{き}かせた
  \jisho{}

\item
  \z{強}{つよ}がる\z{私}{わたし}は\z{臆病}{おくびょう}で
  \jisho{}

  \z{興味}{きょうみ}がないようなふりをしてた
  \jisho{}

  だけど
  \jisho{}

  \z{胸}{むね}を\z{刺}{さ}す\z{痛}{いた}みは\z{増}{ま}してく
  \jisho{}

  ああそうか \z{好}{す}きになるって
  \jisho{}

  こういう\z{事}{こと}なんだね
  \jisho{}

\item
  どうしたい?\z{言}{い}ってごらん
  \jisho{}

  \z{心}{こころ}の\z{声}{こえ}がする
  \jisho{}

  \z{君}{きみ}の\z{隣}{となり}がいい
  \jisho{}

  \z{真実}{しんじつ}は\z{残酷}{ざんこく}だ
  \jisho{}

\item
  \z{言}{い}わなかった
  \jisho{}

  \z{言}{い}えなかった
  \jisho{}

  \z{二度}{にど}と\z{戻}{もど}れない
  \jisho{}

\item
  あの\z{夏}{なつ}の\z{日}{ひ}
  \jisho{}

  きらめく\z{星}{ほし}
  \jisho{}

  \z{今}{いま}でも\z{思}{おも}い\z{出}{だ}せるよ
  \jisho{}

  \z{笑}{わら}った\z{顔}{かお}も
  \jisho{}

  \z{怒}{おこ}った\z{顔}{かお}も
  \jisho{}

  \z{大好}{だいす}きでした
  \jisho{}

  おかしいよね
  \jisho{}

  わかってたのに
  \jisho{}

  \z{君}{きみ}の\z{知}{し}らない
  \jisho{}

  \z{私}{わたし}だけの\z{秘密}{ひみつ}
  \jisho{}

  \z{夜}{よる}を\z{越}{こ}えて
  \jisho{}

  \z{遠}{とお}い\z{思}{おも}い\z{出}{で}の\z{君}{きみ}が
  \jisho{}

  \z{指}{ゆび}をさす
  \jisho{}

  \z{無邪気}{むじゃき}な\z{声}{こえ}で
  \jisho{}

  
}
\end{document}


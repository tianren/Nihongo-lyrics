% !TeX encoding = UTF-8
% !TeX program = LuaLaTeX

%\documentclass[12pt]{article}
\documentclass[14pt]{ltjsarticle}
../macros/dual-lyrics-luamacros.tex
\begin{document}

\lyrics{
\item
  \textbf{まいご \hfill %
    「となりのトトロ」のイメージソング}

\item
  さがしても みつからない まいごの\z{子}{こ}
  \jisho{}

  はなをつみに いったの
  \jisho{}

  ふりむかないで
  \jisho{}

  とんぼを おいかけて いったのかしら
  \jisho{}

  おばけが でてきたら
  \jisho{}

  どうしたらいいの
  \jisho{}

\item
  かくれんぼが だーいすき
  \jisho{}

  なきむしの あまえんぼ
  \jisho{}

  へんじして おねがいだから
  \jisho{}

  きえてしまった ちいさな\z{子}{こ}
  \jisho{}

  どこかしら
  \jisho{}

\item
  おひさまが しずむのに かえらない
  \jisho{}

  こいぬと はらっぱじゅう
  \jisho{}

  はしりまわって
  \jisho{}

  こねこと くさむらで じゃれあって
  \jisho{}

  さよなら できないで
  \jisho{}

  べそをかく わからずや
  \jisho{}

\item
  おにごっこが だーいすき
  \jisho{}

  げんきな きかんぼう
  \jisho{}

  もどってきて おねがいだから
  \jisho{}

  にげてしまった いけない\z{子}{こ}
  \jisho{}

  どこかしら
  \jisho{}

\item
  どうしよう こまったわ
  \jisho{}

  のはらにひとりきり
  \jisho{}

  かえってきて おねがいだから
  \jisho{}

  わたしの だいじな いもうと
  \jisho{}

  どこかしら どこかしら
  \jisho{}

  
}
\end{document}


% !TeX encoding = UTF-8
% !TeX program = LuaLaTeX

%\documentclass[12pt]{article}
\documentclass[14pt]{ltjsarticle}
% !TeX encoding = UTF-8
% !TeX program = LuaLaTeX

% !TeX encoding = UTF-8
% !TeX program = LuaLaTeX

%\documentclass[12pt]{article}

%\usepackage{luatexja-preset}
%\usepackage{fontspec}
\usepackage[no-math]{luatexja-fontspec}
\usepackage{luatexja-ruby}
\usepackage{color}

\topmargin=-0.95in      %
\evensidemargin=0in     %
\oddsidemargin=0in      %
\textwidth=6.5in        %
\textheight=9.75in       %
\headsep=0.25in

\setmainfont{Noto Sans CJK JP}
\setmainjfont{Noto Sans CJK JP}

\makeatletter
\def\dynscriptsize{\check@mathfonts\fontsize{\sf@size}{\z@}\selectfont}
\makeatother
\def\textunderset#1#2{\leavevmode
  \vtop{\offinterlineskip\halign{%
    \hfil##\hfil\cr\strut#2\cr\noalign{\kern-.3ex}
    \hidewidth\dynscriptsize\strut#1\hidewidth\cr}}}
\newcommand{\righttext}[1]{\ifx&#1&\else
  \hspace{1em} \unskip
  \nobreak \hfil \penalty1000 \hfilneg \indent
  \strut\hfill \mbox{\def\z{\zhuyin}\small #1}
  \fi}

\newcommand{\zhuyin}[2]{\ruby{#1}{#2}}
\newcommand{\kanji}[2]{\textunderset{#1}{\textcolor{blue}{#2}}}
\def\jisho{\righttext}
\def\trans{\righttext}

\newcommand{\lyrics}[1]{\begingroup
\def\z{\zhuyin}
\list{}{\leftmargin0em}
\setlength\itemsep{1em}
  #1
\endlist
\endgroup}

\makeatletter
\newcommand{\itemmark}[2]{%
\item[#1.]
\def\@currentlabel{#1}\label{par:#1}
  \expandafter\def\csname firstline@#1\endcsname{#2}%
  #2
}
\newcommand{\itemrepeat}[1]{
  \item[\ref{par:#1}.]
  \csname firstline@#1\endcsname
}
\makeatother


\renewcommand{\jisho}[1]{}

\renewcommand{\lyrics}[1]{\begingroup
\def\z{\zhuyin}
\list{}{\leftmargin0em}
\setlength\itemsep{1em}
  #1
\endlist
\endgroup
\clearpage
\begingroup
\def\z{\kanji}
\list{}{\leftmargin0em}
\setlength\itemsep{1em}
  #1
\endlist
\endgroup}

\begin{document}

\lyrics{
\item
  \textbf{まいご \hfill %
    「となりのトトロ」のイメージソング}

\item
  さがしても みつからない まいごの\z{子}{こ}
  \jisho{}

  はなをつみに いったの
  \jisho{}

  ふりむかないで
  \jisho{}

  とんぼを おいかけて いったのかしら
  \jisho{}

  おばけが でてきたら
  \jisho{}

  どうしたらいいの
  \jisho{}

\item
  かくれんぼが だーいすき
  \jisho{}

  なきむしの あまえんぼ
  \jisho{}

  へんじして おねがいだから
  \jisho{}

  きえてしまった ちいさな\z{子}{こ}
  \jisho{}

  どこかしら
  \jisho{}

\item
  おひさまが しずむのに かえらない
  \jisho{}

  こいぬと はらっぱじゅう
  \jisho{}

  はしりまわって
  \jisho{}

  こねこと くさむらで じゃれあって
  \jisho{}

  さよなら できないで
  \jisho{}

  べそをかく わからずや
  \jisho{}

\item
  おにごっこが だーいすき
  \jisho{}

  げんきな きかんぼう
  \jisho{}

  もどってきて おねがいだから
  \jisho{}

  にげてしまった いけない\z{子}{こ}
  \jisho{}

  どこかしら
  \jisho{}

\item
  どうしよう こまったわ
  \jisho{}

  のはらにひとりきり
  \jisho{}

  かえってきて おねがいだから
  \jisho{}

  わたしの だいじな いもうと
  \jisho{}

  どこかしら どこかしら
  \jisho{}

  
}
\end{document}


% !TeX encoding = UTF-8
% !TeX program = LuaLaTeX

%\documentclass[12pt]{article}
\documentclass[14pt]{ltjsarticle}
% !TeX encoding = UTF-8
% !TeX program = LuaLaTeX

% !TeX encoding = UTF-8
% !TeX program = LuaLaTeX

%\documentclass[12pt]{article}

%\usepackage{luatexja-preset}
%\usepackage{fontspec}
\usepackage[no-math]{luatexja-fontspec}
\usepackage{luatexja-ruby}
\usepackage{color}

\topmargin=-0.95in      %
\evensidemargin=0in     %
\oddsidemargin=0in      %
\textwidth=6.5in        %
\textheight=9.75in       %
\headsep=0.25in

\setmainfont{Noto Sans CJK JP}
\setmainjfont{Noto Sans CJK JP}

\makeatletter
\def\dynscriptsize{\check@mathfonts\fontsize{\sf@size}{\z@}\selectfont}
\makeatother
\def\textunderset#1#2{\leavevmode
  \vtop{\offinterlineskip\halign{%
    \hfil##\hfil\cr\strut#2\cr\noalign{\kern-.3ex}
    \hidewidth\dynscriptsize\strut#1\hidewidth\cr}}}
\newcommand{\righttext}[1]{\ifx&#1&\else
  \hspace{1em} \unskip
  \nobreak \hfil \penalty1000 \hfilneg \indent
  \strut\hfill \mbox{\def\z{\zhuyin}\small #1}
  \fi}

\newcommand{\zhuyin}[2]{\ruby{#1}{#2}}
\newcommand{\kanji}[2]{\textunderset{#1}{\textcolor{blue}{#2}}}
\def\jisho{\righttext}
\def\trans{\righttext}

\newcommand{\lyrics}[1]{\begingroup
\def\z{\zhuyin}
\list{}{\leftmargin0em}
\setlength\itemsep{1em}
  #1
\endlist
\endgroup}

\makeatletter
\newcommand{\itemmark}[2]{%
\item[#1.]
\def\@currentlabel{#1}\label{par:#1}
  \expandafter\def\csname firstline@#1\endcsname{#2}%
  #2
}
\newcommand{\itemrepeat}[1]{
  \item[\ref{par:#1}.]
  \csname firstline@#1\endcsname
}
\makeatother


\renewcommand{\jisho}[1]{}

\renewcommand{\lyrics}[1]{\begingroup
\def\z{\zhuyin}
\list{}{\leftmargin0em}
\setlength\itemsep{1em}
  #1
\endlist
\endgroup
\clearpage
\begingroup
\def\z{\kanji}
\list{}{\leftmargin0em}
\setlength\itemsep{1em}
  #1
\endlist
\endgroup}

\begin{document}

\lyrics{
\item
  \textbf{perfect slumbers \hfill %
    「\z{暦}{こよみ}\z{物語}{ものがたり}」のOP}

\item
  \z{黒}{くろ}い\z{闇}{やみ}の\z{中}{なか}
  \jisho{}

  そっと\z{咲}{さ}いてた
  \jisho{}

  はじめての\z{心}{こころ}
  \jisho{}

  \z{君}{きみ}はきっと\z{知}{し}らない
  \jisho{}

\item
  \z{差}{さ}し\z{伸}{の}べてくれた\z{手}{て}を
  \jisho{}

  つかめずにいる\z{私}{わたし}は
  \jisho{}

  あと\z{少}{すこ}しもう\z{少}{すこ}し
  \jisho{}

  \z{君}{きみ}にもっと\z{求}{もと}めている
  \jisho{}

\item
  それはひとすじの\z{光}{ひかり}
  \jisho{}

\item
  \z{心}{こころ}の\z{行方}{ゆくえ}を
  \jisho{}

  \z{今}{いま}は まだ\z{知}{し}りたくない
  \jisho{}

\item
  \z{黒}{くろ}い\z{闇}{やみ}の\z{中}{なか}
  \jisho{}

  そっと\z{咲}{さ}いてた
  \jisho{}

  はじめての\z{心}{こころ}
  \jisho{}

  \z{今日}{きょう}も\z{云}{い}えないままで
  \jisho{}

\item
  あきらめたふりをして
  \jisho{}

  \z{期待}{きたい}してしまっている
  \jisho{}

  \z{君}{きみ}になら\z{君}{きみ}となら
  \jisho{}

  だからこそ\z{怯}{おび}えている
  \jisho{}

\item
  それは あたたかな\z{光}{ひかり}
  \jisho{}

\item
  \z{多分}{たぶん}\z{今}{いま}はまだきっと\z{夜明}{よあ}け\z{前}{まえ}
  \jisho{}

  \z{君}{きみ}のとなり\z{微睡}{まどろ}みながら\z{明日}{あした}を\z{待}{ま}ってる
  \jisho{}

\item
  \z{差}{さ}し\z{伸}{の}べてくれた\z{手}{て}を
  \jisho{}

  つかめずにいる\z{私}{わたし}は
  \jisho{}

  あと\z{少}{すこ}しもう\z{少}{すこ}し
  \jisho{}

  \z{君}{きみ}にばかり\z{求}{もと}めている
  \jisho{}

\item
  それはひとすじの\z{光}{ひかり}
  \jisho{}

\item
  はじめての\z{心}{こころ}
  \jisho{}

  \z{君}{きみ}に \z{気付}{きづ}いてほしくて
  \jisho{}

  
}
\end{document}


% !TeX encoding = UTF-8
% !TeX program = LuaLaTeX

%\documentclass[12pt]{article}
\documentclass[14pt]{ltjsarticle}
../macros/dual-lyrics-luamacros.tex
\begin{document}

\lyrics{
\item
  \textbf{perfect slumbers \hfill %
    「\z{暦}{こよみ}\z{物語}{ものがたり}」のOP}

\item
  \z{黒}{くろ}い\z{闇}{やみ}の\z{中}{なか}
  \jisho{}

  そっと\z{咲}{さ}いてた
  \jisho{}

  はじめての\z{心}{こころ}
  \jisho{}

  \z{君}{きみ}はきっと\z{知}{し}らない
  \jisho{}

\item
  \z{差}{さ}し\z{伸}{の}べてくれた\z{手}{て}を
  \jisho{}

  つかめずにいる\z{私}{わたし}は
  \jisho{}

  あと\z{少}{すこ}しもう\z{少}{すこ}し
  \jisho{}

  \z{君}{きみ}にもっと\z{求}{もと}めている
  \jisho{}

\item
  それはひとすじの\z{光}{ひかり}
  \jisho{}

\item
  \z{心}{こころ}の\z{行方}{ゆくえ}を
  \jisho{}

  \z{今}{いま}は まだ\z{知}{し}りたくない
  \jisho{}

\item
  \z{黒}{くろ}い\z{闇}{やみ}の\z{中}{なか}
  \jisho{}

  そっと\z{咲}{さ}いてた
  \jisho{}

  はじめての\z{心}{こころ}
  \jisho{}

  \z{今日}{きょう}も\z{云}{い}えないままで
  \jisho{}

\item
  あきらめたふりをして
  \jisho{}

  \z{期待}{きたい}してしまっている
  \jisho{}

  \z{君}{きみ}になら\z{君}{きみ}となら
  \jisho{}

  だからこそ\z{怯}{おび}えている
  \jisho{}

\item
  それは あたたかな\z{光}{ひかり}
  \jisho{}

\item
  \z{多分}{たぶん}\z{今}{いま}はまだきっと\z{夜明}{よあ}け\z{前}{まえ}
  \jisho{}

  \z{君}{きみ}のとなり\z{微睡}{まどろ}みながら\z{明日}{あした}を\z{待}{ま}ってる
  \jisho{}

\item
  \z{差}{さ}し\z{伸}{の}べてくれた\z{手}{て}を
  \jisho{}

  つかめずにいる\z{私}{わたし}は
  \jisho{}

  あと\z{少}{すこ}しもう\z{少}{すこ}し
  \jisho{}

  \z{君}{きみ}にばかり\z{求}{もと}めている
  \jisho{}

\item
  それはひとすじの\z{光}{ひかり}
  \jisho{}

\item
  はじめての\z{心}{こころ}
  \jisho{}

  \z{君}{きみ}に \z{気付}{きづ}いてほしくて
  \jisho{}

  
}
\end{document}


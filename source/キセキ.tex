% !TeX encoding = UTF-8
% !TeX program = LuaLaTeX

%\documentclass[12pt]{article}
\documentclass[14pt]{ltjsarticle}
../macros/dual-lyrics-luamacros.tex
\begin{document}

\lyrics{
\item
  \textbf{キセキ \hfill GReeeeN\\
    \hfill「からかい上手の高木さん」のED}

\item
  \z{明日}{あした}、\z{今日}{きょう}よりも\z{好}{す}きになれる \z{溢}{あふ}れる\z{想}{おも}いが\z{止}{と}まらない
  \jisho{}

  \z{今}{いま}もこんなに\z{好}{す}きでいるのに \z{言葉}{ことば}に\z{出来}{でき}ない
  \jisho{}

\item
  \z{君}{きみ}のくれた\z{日々}{ひび}が\z{積}{つ}み\z{重}{かさ}なり \z{過}{す}ぎ\z{去}{さ}った\z{日々2人歩}{ひびふたりある}いた『\z{軌跡}{きせき}』
  \jisho{}

  \z{僕}{ぼく}らの\z{出逢}{であ}いがもし\z{偶然}{ぐうぜん}ならば? \z{運命}{うんめい}ならば? \z{君}{きみ}に\z{巡}{めぐ}り\z{合}{あ}えた それって『\z{奇跡}{きせき}』
  \jisho{}

\item
  \z{2人寄}{ふたりよ}り\z{添}{そ}って\z{歩}{ある}いて \z{永久}{とわ}の\z{愛}{あい}を\z{形}{かたち}にして
  \jisho{}

  
いつまでも\z{君}{きみ}の\z{横}{よこ}で \z{笑}{わら}っていたくて
  \jisho{}

  
アリガトウや \z{Ah}{アー} \z{愛}{あい}してるじゃまだ\z{足}{た}りないけど
  \jisho{}

  
せめて\z{言}{い}わせて「\z{幸}{しあわ}せです」と
  \jisho{}

\item
  
いつも\z{君}{きみ}の\z{右}{みぎ}の\z{手}{て}の\z{平}{ひら}を ただ\z{僕}{ぼく}の\z{左}{ひだり}の\z{手}{て}の\z{平}{ひら}が
  \jisho{}

  
そっと\z{包}{つつ}んでくそれだけで ただ\z{愛}{あい}を\z{感}{かん}じていた
  \jisho{}

\item
  \z{日々}{ひび}の\z{中}{なか}で \z{小}{ちい}さな\z{幸}{しあわ}せ \z{見}{み}つけ\z{重}{かさ}ね ゆっくり\z{歩}{ある}いた『\z{軌跡}{きせき}』
  \jisho{}

  \z{僕}{ぼく}らの\z{出逢}{であ}いは\z{大}{おお}きな\z{世界}{せかい}で \z{小}{ちい}さな\z{出来事}{できごと} \z{巡}{めぐ}り\z{合}{あ}えた それって『\z{奇跡}{きせき}』
  \jisho{}

\item
  
うまく\z{行}{い}かない\z{日}{ひ}だって \z{2人}{ふたり}で\z{居}{い}れば\z{晴}{は}れだって!
  \jisho{}

  \z{強}{つよ}がりや\z{寂}{さび}しさも \z{忘}{わす}れられるから
  \jisho{}

  \z{僕}{ぼく}は\z{君}{きみ}でなら \z{僕}{ぼく}でいれるから!
  \jisho{}

  
だからいつも そばに\z{居}{い}てよ『\z{愛}{いと}しい\z{君}{きみ}へ』
  \jisho{}

\item
  \z{2人}{ふたり}フザけあった\z{帰}{かえ}り\z{道}{みち} それも\z{大切}{たいせつ}な\z{僕}{ぼく}らの\z{日々}{ひび}
  \jisho{}

  
「\z{想}{おも}いよ\z{届}{とど}け!!!」と\z{伝}{つた}えた\z{時}{とき}に \z{初}{はじ}めて\z{見}{み}せた\z{表情}{ひょうじょう}の\z{君}{きみ}
  \jisho{}

  \z{少}{すこ}し\z{間}{ま}が\z{空}{あ}いて \z{君}{きみ}がうなずいて \z{僕}{ぼく}らの\z{心}{こころ} \z{満}{み}たされてく\z{愛}{あい}で
  \jisho{}

  \z{僕}{ぼく}らまだ\z{旅}{たび}の\z{途中}{とちゅう}で またこれから\z{先}{さき}も \z{何十年続}{なんじゅうねんつづ}いていけるような\z{未来}{みらい}ヘ
  \jisho{}

\item
  \z{例}{たと}えばほら \z{明日}{あした}を\z{見失}{みうしな}いそうに \z{僕}{ぼく}らなったとしても、、、
  \jisho{}

\item
  \z{2人寄}{ふたりよ}り\z{添}{そ}って\z{歩}{ある}いて \z{永久}{とわ}の\z{愛}{あい}を\z{形}{かたち}にして
  \jisho{}

  
いつまでも\z{君}{きみ}の\z{横}{よこ}で \z{笑}{わら}っていたくて
  \jisho{}

  
アリガトウや \z{Ah}{アー} \z{愛}{あい}してるじゃまだ\z{足}{た}りないけど
  \jisho{}

  
せめて\z{言}{い}わせて「\z{幸}{しあわ}せです」と
  \jisho{}

\item
  
うまく\z{行}{い}かない\z{日}{ひ}だって \z{2人}{ふたり}で\z{居}{い}れば\z{晴}{は}れだって!
  \jisho{}

  \z{喜}{よろこ}びや\z{悲}{かな}しみも \z{全}{すべ}て\z{分}{わ}け\z{合}{あ}える
  \jisho{}

  \z{君}{きみ}が\z{居}{い}るから \z{生}{い}きていけるから!
  \jisho{}

  
だからいつも そばに\z{居}{い}てよ『\z{愛}{いと}しい\z{君}{きみ}へ』\z{最後}{さいご}の\z{一秒}{いちびょう}まで
  \jisho{}

\item
  \z{明日}{あした}、\z{今日}{きょう}より\z{笑顔}{えがお}になれる \z{君}{きみ}が\z{居}{い}るだけで そう\z{思}{おも}えるから
  \jisho{}

  \z{何十年}{なんじゅうねん} \z{何百年}{なんびゃくねん} \z{何千年}{なんぜんねん} \z{時}{とき}を\z{超}{こ}えよう \z{君}{きみ}を\z{愛}{あい}してる
  \jisho{}

}
\end{document}


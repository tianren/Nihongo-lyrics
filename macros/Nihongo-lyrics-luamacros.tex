% !TeX encoding = UTF-8
% !TeX program = LuaLaTeX

%\documentclass[12pt]{article}

%\usepackage{luatexja-preset}
%\usepackage{fontspec}
\usepackage[no-math]{luatexja-fontspec}
\usepackage{luatexja-ruby}
\usepackage{color}

\topmargin=-0.95in      %
\evensidemargin=0in     %
\oddsidemargin=0in      %
\textwidth=6.5in        %
\textheight=9.75in       %
\headsep=0.25in

\setmainfont{Noto Sans CJK JP}
\setmainjfont{Noto Sans CJK JP}

\makeatletter
\def\dynscriptsize{\check@mathfonts\fontsize{\sf@size}{\z@}\selectfont}
\makeatother
\def\textunderset#1#2{\leavevmode
  \vtop{\offinterlineskip\halign{%
    \hfil##\hfil\cr\strut#2\cr\noalign{\kern-.3ex}
    \hidewidth\dynscriptsize\strut#1\hidewidth\cr}}}
\newcommand{\righttext}[1]{\ifx&#1&\else
  \hspace{1em} \unskip
  \nobreak \hfil \penalty1000 \hfilneg \indent
  \strut\hfill \mbox{\def\z{\zhuyin}#1}
  \fi}
\makeatletter
\newcommand{\rightpara}[1]{\begingroup\ifx&#1&\else%
  \hspace{1em}\unskip%
  \nobreak \hfil \penalty1000 \hfilneg \indent%
  % \strut%
  \hfill%
  \setlength{\parfillskip}{0pt}%
  % \raggedleft%
  % \let\\\@centercr
  % \rightskip\z@skip%
  \leftskip\@flushglue
  % \parindent\z@\parfillskip\z@skip
  \mbox{}\linebreak[0] #1\par
  \fi\endgroup%
}
\makeatother

\newcommand{\zhuyin}[2]{\ruby{#1}{#2}}
\newcommand{\kanji}[2]{\textunderset{#1}{\textcolor{blue}{#2}}}
\newcommand{\hide}[1]{}
\makeatletter
\newcommand{\jisho}[1]{\ifx&#1&\else\rightpara{\small%
  \@for \el:=#1\do{\mbox{\el.} }%
  % \par
  }\fi}
\makeatother
\newcommand{\trans}[1]{\righttext{\small #1}}

\newcommand{\lyrics}[1]{\begingroup
\def\z{\zhuyin}
\list{}{\leftmargin0em}
\setlength\itemsep{1em}
  #1
\endlist
\endgroup}

%
% Section reusage
%
\makeatletter
\newcommand{\itemmark}[3]{%
\item[#1.]
\def\@currentlabel{#1}\label{par:#1}
  \expandafter\def\csname firstline@#1\endcsname{#2}%
  \expandafter\def\csname remaininglines@#1\endcsname{#3}%
  #2
  #3
}
\newcommand{\itemfirstline}[1]{\csname firstline@#1\endcsname}
\newcommand{\itemcontext}[1]{\csname firstline@#1\endcsname
  \csname remaininglines@#1\endcsname}

\newcommand{\itemref}[2][.]{\ref{par:#2}#1}
\newcommand{\itemsref}[2][.]{%
  \def\itemref@nextspace{}%
  \@for \el:=#2\do{\itemref@nextspace\itemref[#1]{\el}\def\itemref@nextspace{ }}%
}
\newcommand{\itemseen}[1]{\item[\itemsref{#1}]}

\newcommand{\itemrepeat}[1]{\itemseen{#1}\itemcontext{#1}}
\newcommand{\itemsrepeat}[1]{
  \@for \el:=#1\do{\itemrepeat{\el}}
}
\newcommand{\itemrepeatfirstline}[1]{\itemseen{#1}%
  \def\operation{\itemfirstline}%
  \@for \el:=#1\do{\operation{\el}\def\operation{\hide}}%
  }
\makeatother

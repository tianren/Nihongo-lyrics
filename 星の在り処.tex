% !TeX encoding = UTF-8
% !TeX program = LuaLaTeX

%\documentclass[12pt]{article}
\documentclass[14pt]{extreport}
../macros/dual-lyrics-luamacros.tex
\begin{document}

\lyrics{
\item
  \textbf{星の在り処 \hfill 「空の軌跡FC」のED}

\item
  \z{君}{きみ}の\z{影}{かげ} \z{星}{ほし}のように \z{朝}{あさ}に\z{溶}{と}けて\z{消}{き}えていく
  \jisho{}

  \z{行}{い}き\z{先}{さき}を\z{失}{な}くしたまま \z{想}{おも}いは\z{溢}{あふ}れてくる
  \jisho{}

\item
  \z{強}{つよ}さにも\z{弱}{よわ}さにも この\z{心}{こころ}は\z{向}{む}き\z{合}{あ}えた
  \jisho{}

  \z{君}{きみ}とならどんな\z{明日}{あす}が \z{来}{き}ても\z{怖}{こわ}くないのに
  \jisho{}

\item
  \z{二人}{ふたり}\z{歩}{ある}いた\z{時}{とき}を \z{信}{しん}じていてほしい
  \jisho{}

\item
  \z{真実}{しんじつ}も\z{嘘}{うそ}もなく \z{夜}{よる}が\z{明}{あ}けて\z{朝}{あさ}が\z{来}{く}る
  \jisho{}

  \z{星空}{ほしぞら}が\z{朝}{あさ}に\z{溶}{と}けても \z{君}{きみ}の\z{輝}{かがや}きはわかるよ
  \jisho{}

\item
  さよならを\z{知}{し}らないで \z{夢見}{ゆめみ}たのは \z{一人}{ひとり}きり
  \jisho{}

  あの\z{頃}{ころ}の \z{君}{きみ}の\z{目}{め}には \z{何}{なに}が\z{映}{うつ}っていたの?
  \jisho{}

\item
  \z{二人}{ふたり}つないだ\z{時}{とき}を \z{誰}{だれ}も\z{消}{け}せはしない
  \jisho{}

\item
  \z{孤独}{こどく}とか\z{痛}{いた}みとか どんな\z{君}{きみ}も\z{感}{かん}じたい
  \jisho{}

  もう\z{一度}{いちど} \z{見}{み}つめ\z{合}{あ}えれば \z{願}{ねが}いはきっと\z{叶}{かな}う
  \jisho{}

\item
  \z{夜明}{よあ}け\z{前}{まえ} まどろみに \z{風}{かぜ}が\z{頬}{ほお}を \z{流}{なが}れていく
  \jisho{}

  \z{君}{きみ}の\z{声}{こえ} \z{君}{きみ}の\z{香}{かお}りが \z{全}{すべ}てを\z{包}{つつ}んで\z{満}{み}ちていく
  \jisho{}

\item
  \z{思}{おも}い\z{出}{で}を\z{羽}{は}ばたかせ \z{君}{きみ}の\z{空}{そら}へ\z{舞}{ま}い\z{上}{あ}がる
  \jisho{}

  \z{星空}{ほしぞら}が\z{朝}{あさ}に\z{溶}{と}けても \z{君}{きみ}の\z{輝}{かがや}きはわかるよ
  \jisho{}

\item
  \z{愛}{あい}してる ただそれだけで \z{二人}{ふたり}はいつかまた\z{会}{あ}える
  \jisho{}


}

\end{document}
